\chapter{Introdução}

Inteligência artificial é um campo da ciência da computação que estuda
``agentes inteligentes'', que de certa forma percebem o ambiente à sua volta
e tomam ações tendo como objetivo maximizar a chance de sucesso de alguma
tarefa específica. Uma utilização muito comum nos dias de hoje é a de
inteligência artificial com o objetivo de criar um agente capaz de competir
com humanos em jogos com alto nível de estratégia, como Xadrez\footnote{inserir referência
na bibliografia} e Go\footnote{disso também}.
Decidimos então estudar inteligência artificial para a realização deste
trabalho com a ideia de modelar um outro jogo de estratégia,
\textbf{Magic: The Gathering}.

\textit{Magic} foi lançado em 1993, introduzindo o conceito de trading card
game. Com o sucesso do jogo e popularização dos jogos digitais, eventualmente
nasceram versões digitais do jogo para computadores e videogames, e com isso
nasceu a necessidade de agentes que jogassem contra os jogadores. Atualmente
há várias versões digitais de Magic, mas nenhuma tem uma inteligência artificial
boa o suficiente para se provar desafiadora contra jogadores experientes, uma vez
que a cada ação há uma grande quantidade de ações possíveis e elementos como
blefe envolvidos. Nossa intenção é entender a complexidade da representação do
jogo e criar uma plataforma para jogar Magic que possibilite a implementação de
um agente de inteligência artificial.

Na próxima seção iremos introduzir alguns conceitos e regras básicas do jogo,
de modo a possibilitar a familiarização do leitor com \textbf{Magic}, facilitando
a compreensão do restante do trabalho.

\section{Conceitos básicos}
Um jogo usual de \textit{Magic: the Gathering} conta com dois jogadores munidos
de um baralho de 60 cartas cada, ambos começando com 20 pontos de vida, sendo
que o objetivo é reduzir o total de pontos de vida do oponente a 0. Para tanto,
é preciso usar as cartas disponíveis na mão, que podem representar feitiços,
criaturas ou terrenos (existem outros tipos de cartas, mas para nossa implementação
iremos focar nesses três).

\subsection{Cartas}

Nesta seção explicamos a estrutura de uma carta e as diferenças entre os três tipos de carta que utilizamos no projeto. Antes de tudo, convém definir alguns termos utilizados no jogo:

\begin{itemize}
  \item \textbf{Jogar} uma carta é um termo genérico para ``retirar uma carta da sua mão e aplicar seus efeitos no jogo''. Dependendo do tipo de uma carta, quando esta é ``jogada'', será colocada no \textit{campo de batalha} (termo próprio para a mesa de jogo) ou no \textit{cemitério} (termo próprio para a pilha de descarte) do jogador. Uma carta na mesa é chamada \textit{permanente}.
  \item \textbf{Virar} (em inglês, \textit{tap}) uma permanente significa fisicamente dispô-la horizontalmente (normalmente uma carta está na posição vertical). Esta ação representa uma espécie de ação das permenentes. Quando o turno começa, o jogador ativo \textit{desvira} suas permanentes (ou seja, as dispõe em posição vertical), portanto, usualmente, virar uma permanente é um movimento que acontecerá em até uma vez (por permanente) durante cada turno.
\end{itemize}

A seguir, examinaremos a estrutura de uma carta.

\begin{figure}[!ht]
    \centering
    \includegraphics[width=0.7\textwidth]{picstcc/angelnumbers.png}
    \caption{Angel of Mercy e seus principais atributos.}
    \label{cardinfo}
\end{figure}

\begin{enumerate}
  \item \textbf{Nome} da carta.
  \item \textbf{Custo de mana}. ``Mana'' é o recurso necessário em \textit{Magic} para jogar cartas que não sejam terrenos. Os símbolos da caixa representam a quantidade e tipo de mana necessários para jogar Angel of Mercy.
  $\vcenter{\hbox{\includegraphics[scale=0.015]{picstcc/4.png}}}$$\vcenter{\hbox{\includegraphics[scale=0.015]{picstcc/W.png}}}$, então, significa que, para jogar a carta é preciso 4 ``manas'' de qualquer tipo e uma ``mana'' branca, totalizando 5. Mana é geralmente de um dos 5 tipos -- chamados \textit{cores} -- possíveis: Branco ($\vcenter{\hbox{\includegraphics[scale=0.015]{picstcc/W.png}}}$), Azul, ($\vcenter{\hbox{\includegraphics[scale=0.015]{picstcc/U.png}}}$), Preto ($\vcenter{\hbox{\includegraphics[scale=0.015]{picstcc/B.png}}}$), Vermelho ($\vcenter{\hbox{\includegraphics[scale=0.015]{picstcc/R.png}}}$) e Verde ($\vcenter{\hbox{\includegraphics[scale=0.015]{picstcc/G.png}}}$). O custo ainda determina a \textit{cor} da própria carta: Angel of Mercy requer apenas mana branca, portanto é uma carta branca.
  \item \textbf{Tipos} de cartas são escritos nesta caixa. Como já mencionado, o tipo de Angel of Mercy é Criatura. Além disso, cartas -- quase sempre criaturas -- também podem ter \textit{subtipos} indicando características da criatura, no caso, um Anjo. Isto é relevante para alguns efeitos do jogo, mas nenhum deles é abordado no projeto.
  \item \textbf{Habilidades} definem os efeitos próprios das cartas. Neste trabalho, usamos apenas criaturas com \textit{habilidades de combate}, características de cada criatura que definem seu comportamento na etapa de combate, e habilidades do tipo ``ao entrar em jogo'', desencadeadas quando a carta se torna uma permanente. \\ No caso, Angel of Mercy tem instâncias dos dois tipos: \textit{Flying} (em portugês, Voar), que limita as interações do oponente durante a etapa de Combate, e sua segunda habilidade, que concede a seu controlador um bônus de 3 pontos de vida. Existem, também, criaturas sem habilidades (porém não existem feitiços sem habilidades). Algumas cartas contêm explicações, em parênteses, de efeitos comuns (como Voar).
  \item \textit{\textbf{Flavor text}} é um texto que ambienta a carta dentro de alguma temática. Pode ser uma fala, um comentário, até uma pequena história. \textit{Flavor texts} são escritos em itálico e não influenciam em nada as regras do jogo.
  \item Esta caixa contém os atributos mais importantes das criaturas (e presentes apenas neste tipo de carta): \textbf{poder e resistência}. $X/Y$ representa uma criatura com poder $X$ e resistência $Y$, ou seja, na etapa de combate, esta criatura causa $X$ de dano total e, a qualquer momento, se a criatura tiver dano marcado $d$ tal que $d \ge Y$, a criatura é destruída.
\end{enumerate}
Os \textbf{terrenos} citados são responsáveis por gerar mana para jogar cartas com custo. Para jogar um terreno, o jogador ativo simplesmente o retira da sua mão e coloca em campo (podendo fazer isso no máximo uma vez por turno). Neste trabalho usamos apenas terrenos com o supertipo \textit{Básico}.\footnote{Geralmente, em \textit{Magic}, um baralho tem a limitação de 4 da mesma carta, \textit{exceto} cartas com este supertipo. Até o momento, existem apenas terrenos com o supertipo.} A cor da mana gerada pelo terreno está ligada a seu subtipo e seu símbolo associado também aparece na carta. Todos os terrenos básicos têm a habilidade ``$\vcenter{\hbox{\includegraphics[scale=0.015]{picstcc/T.png}}}$: adicione $C$ à sua reserva de mana.'', onde $C \in \left\{ \vcenter{\hbox{\includegraphics[scale=0.015]{picstcc/W.png}}}, \vcenter{\hbox{\includegraphics[scale=0.015]{picstcc/U.png}}}, \vcenter{\hbox{\includegraphics[scale=0.015]{picstcc/B.png}}}, \vcenter{\hbox{\includegraphics[scale=0.015]{picstcc/R.png}}}, \vcenter{\hbox{\includegraphics[scale=0.015]{picstcc/G.png}}}\right\}$, não escrita na carta.
\begin{figure}[!h]
  \centering
  \includegraphics[width=0.7\textwidth]{picstcc/basiclands.png}
  \caption{Os cinco terrenos básicos. Em sentido horário: Planície, Ilha, Pântano, Montanha, Floresta.}
  \label{basiclands}
\end{figure}
\vskip1ex

Assim, a rotina para jogar uma carta com determinado custo é virar o número de terrenos necessários de cada tipo, remover a carta da mão e aplicar seus efeitos (caso seja uma Criatura, é colocada em campo e se torna uma Permanente, caso seja um Feitiço, seus efeitos são aplicados e é colocada no cemitério). No caso de Angel of Mercy, o jogador deve virar 4 terrenos quaisquer e uma Planície para jogá-la. Assim, que o anjo entrar em jogo, seu efeito é desencadeado Apesar de ser possível (e usual) usar mais de uma cor em um baralho, considerados no trabalho são de uma cor só.

\begin{figure}[!h]
  \centering
  \includegraphics[width=0.5\textwidth]{picstcc/volcanicfull.png}
  \caption{O Feitiço ``Volcanic Hammer'' (Martelo Vulcânico)}
  \label{volcanicsorcery}
\end{figure}

A figura \ref{volcanicsorcery} contém, enfim, um exemplo de Feitiço. A carta possui todas as características de uma carta de Criatura, exceto pelo poder e resistência. Para jogar Volcanic Hammer, além de virar o número de terrenos necesário (ou seja, uma Montanha e um terreno qualquer para pagar o custo de $\vcenter{\hbox{\includegraphics[scale=0.015]{picstcc/1.png}}} \vcenter{\hbox{\includegraphics[scale=0.015]{picstcc/R.png}}}$), o jogador ativo deve eleger um \textbf{alvo} para o efeito da carta. O próprio efeito descreve os alvos legais -- no caso, o alvo escolhido pode ser qualquer criatura em campo \textit{ou} qualquer jogador. Alguns efeitos tem seus alvos restritos, por exemplo, a criaturas do oponente ou somente a jogadores.

\vskip1ex

\label{gamestructure}

No começo do jogo é decidido aleatoriamente quem será o jogador inicial,
e então os dois jogadores compram uma mão inicial de sete cartas.
Antes do jogo propriamente dito começar, os jogadores podem optar por tomar uma
ação chamada \textit{mulligan}, que consiste em rejeitar a mão inicial, embaralhá-la
de volta com o restante dos cards e comprar uma nova mão inicial, com uma carta
a menos. Pode-se então repetir o processo até que cada jogador esteja satisfeito
com a mão inicial ou até o jogador realizar um mulligan com apenas uma carta na
mão (resultando em uma mão de zero cartas, onde não há mais a possibilidade de
realizar mulligan). Uma vez que os dois jogadores tiverem escolhido manter uma
mão inicial, cada jogador que realizou pelo menos um mulligan olha a carta do
topo de seu \textit{deck} (como é chamado o baralho) e decide se quer colocá-la
no fundo.

\vskip1ex

O jogo então começa, com os jogadores alternando entre turnos, onde o
jogador que ``controla o turno'' é chamado de \textit{jogador ativo},
com a seguinte estrutura, simplificada:

\begin{itemize}
    \item\textbf{Início do turno}: Permanentes do jogador ativo são
desviradas. Jogador ativo compra uma carta de seu deck.
    \item\textbf{Primeira Fase Principal}: Jogador ativo pode jogar as
cartas da mão. Termina quando o jogador ativo decide \textit{passar} o turno.
    \item \textbf{Combate}: O combate é a maneira principal de um jogador ganhar o jogo, pois envolve a tentativa de diminuição dos pontos de vida de seu oponente através de ataques de suas criaturas. Será melhor explicado na próxima seção.
    \item \textbf{Segunda Fase Principal}: Igual à primeira Fase Principal.
    \item \textbf{Fase final}: Todas as criaturas tem seu dano marcado revertido para 0. Caso o jogador tenha mais de sete cartas na mão, deve descartar (colocar no \textit{cemitério}) até ter uma mão de sete cartas.
\end{itemize}

A estrutura acima se repete até o jogo terminar, o que acontece
geralmente quando algum jogador chega a 0 pontos de vida,
mas também pode acontecer de outras maneiras como, por exemplo, se o
baralho de um jogador acabar.

\subsection{Exemplo de Fases Principais}

Suponhamos que o jogador ativo tem 5 Planícies desviradas e 20 pontos de vida. Na sua primeira Fase Principal, joga seu Angel of Mercy (virando as 5 planícies para adicionar $\vcenter{\hbox{\includegraphics[scale=0.015]{picstcc/W.png}}} \vcenter{\hbox{\includegraphics[scale=0.015]{picstcc/W.png}}}\vcenter{\hbox{\includegraphics[scale=0.015]{picstcc/W.png}}}\vcenter{\hbox{\includegraphics[scale=0.015]{picstcc/W.png}}}\vcenter{\hbox{\includegraphics[scale=0.015]{picstcc/W.png}}}$ à sua reserva de mana), que tem seu efeito desencadeado levando-o a 23 pontos de vida. Decide, então, passar o turno para a fase de Combate. Como não tem nenhuma criatura apta a atacar (melhor explicado na próxima subseção), a fase de Combate termina e começa a segunda Fase Principal. O jogador ativo decide passar a segunda fase principal. No próximo turno, o jogador ativo é trocado, compra uma carta e inicia sua primeira Fase Principal. O (novo) jogador ativo tem duas montanhas e decide jogar Volcanic Hammer alvejando Angel of Mercy (e para fazê-lo, vira as duas montanhas, gerando $\vcenter{\hbox{\includegraphics[scale=0.015]{picstcc/1.png}}} \vcenter{\hbox{\includegraphics[scale=0.015]{picstcc/R.png}}}$). Como o anjo tem 3 de resistência e o Feitiço lhe causa 3 dano, a permanente é destruída.

\subsection{Combate}

A fase de Combate é a principal maneira do jogador ativo diminuir o total de vida de seu oponente. Isto é realizado ao \textbf{atacar} com suas criaturas. O jogador defensor, como é chamado o oponente do jogador ativo, tem a chance de \textbf{bloquear} os ataques com as suas próprias criaturas. A maioria dos jogos de \textit{Magic} termina durante esta fase.

\begin{enumerate}
  \item Jogador ativo \textit{declara atacantes} -- escolhe
quais de suas criaturas irão atacar seu oponente. Uma criatura só pode atacar se 1) estiver \textbf{desvirada} (representada na posição vertical) e 2) não estiver \textit{enjoada} (termo que significa que a criatura entrou em jogo neste turno -- uma criatura só pode atacar a partir do turno seguinte ao turno que entrou no campo de batalha). Uma vez que o jogador decidiu com quais criaturas atacar, deve \textbf{virar} as criaturas declaradas para indicar que estas estão atacando. Se o jogador ativo não tiver criaturas aptas a atacar (ou decidir não atacar com nenhuma), a fase de Combate termina.
\\ Como exemplo
    \item O jogador defensor \textit{declara bloqueadores} -- escolhe quais de suas criaturas irão
bloquear as criaturas atacantes. Mais uma vez, apenas criaturas \textbf{desviradas} podem bloquear. Diferentemente das criaturas atacantes, porém, criaturas bloqueadoras \textbf{não} são viradas. Algumas habilidades restringem como criaturas podem bloquear: uma criatura atacante com Voar só pode ser bloqueada por criaturas com a habilidade Voar ou Alcance (\textit{Reach}).
    \item Caso alguma criatura atacante esteja bloqueada
por mais de uma criatura, o jogador ativo decide, então, a ordem com que o dano será
atribuído a cada criatura bloqueadora.
    \item Por fim, há a atribuição de dano. Cada criatura atacante causa dano igual a seu poder às criaturas que a bloqueiam e, caso, não esteja bloqueada, causa dano igual ao seu poder ao jogador defensor. Cada criatura bloqueadora causa dano igual ao seu poder à criatura que bloqueou.
\end{enumerate}

\subsection{Exemplo de Combate}

Suponhamos que o jogador ativo tenha três criaturas desviradas e não-enjoadas no início da fase de combate e seu oponente tenha duas criaturas desviradas (portanto, aptas a bloquear) e 20 pontos de vida.

\begin{figure}[!h]
  \centering
  \includegraphics[width=0.8\textwidth]{picstcc/att1.png}
  \caption{Criaturas do jogador ativo: dois ``Grizzly Bears'' (Ursos Cinzentos) em uma ``Centaur Courser'' (Centaura-Caçadora).}
  \label{beginattack}
\end{figure}

\newpage

\begin{figure}[!h]
  \centering
  \includegraphics[width=0.55\textwidth]{picstcc/blk1.png}
  \caption{Criaturas do jogador defensor: dois ``Walking Corpse'' (Cadáver Ambulante).}
  \label{beginblock}
\end{figure}

Supondo que o jogador ativo decida atacar com Centaur Courser e um Grizzly Bears, deve então virá-los.

\begin{figure}[!h]
  \centering
  \includegraphics[width=0.8\textwidth]{picstcc/att2.png}
  \caption{Ataque submetido pelo jogador ativo.}
  \label{declaredattackers}
\end{figure}

\newpage

Por fim, o jogador defensor deve declarar quais de suas criaturas bloquearão a ofensiva. Suponhamos, então, que a escolha seja apenas bloquear Grizzly Bears com um Walking Corpse.

\begin{figure}[!h]
  \centering
  \includegraphics[width=0.8\textwidth]{picstcc/blk2.png}
  \caption{Representação dos bloqueios.}
  \label{declaredblockers}
\end{figure}

Assim, no final do combate, Grizzly Bears causa 2 de dano a Walking Corpse, que por sua vez também lhe causa 2 de dano. Ambas as criaturas são destruídas. Centaur Courser causa 3 de dano ao jogador defensor. O jogador defensor termina 17 de vida e um Walking Corpse desvirado, enquanto o jogador ativo tem 20 de vida e um Centaur Courser virado.
