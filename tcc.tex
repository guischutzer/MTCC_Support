\documentclass{book}
\usepackage[utf8]{inputenc}
\usepackage[portuguese]{babel}
\usepackage{graphicx}
\usepackage{fullpage}
\usepackage{wrapfig}
\usepackage{tikz}
\usepackage{amsmath}

\title{Inteligência Artificial em Magic: The Gathering\\
        Proposta}
\author{Guilherme Souto Schützer - 8658544 \\
        Tomás Bezerra Marcondes Paim - 7157602}
\date{28 de abril de 2017}

\begin{document}

\maketitle

\chapter{Introdução}

\section{Regras}
\textbf{Magic: The Gathering} é um jogo de cartas criado em 1993 por
Richard Garfield que introduziu o conceito moderno de \textit{trading
card game} (jogo de cartas colecionáveis). Com um acervo de mais de 15
mil cartas diferentes, os jogadores devem criar baralhos de 60 cartas
(normalmente podendo haver repetição de até quatro cartas iguais) e
competir, normalmente em jogos um contra um.
\par Durante o jogo, os jogadores têm que lidar com informações
conhecidas (as cartas já jogadas previamente e as cartas em suas mãos) e
informações desconhecidas (as cartas na mão de seu oponente e a ordem
das cartas em seu baralho), o que faz com que seja praticamente
impossível ter informação perfeita. Além disso, ambos os jogadores podem
agir a praticamente qualquer momento dentro de um turno, o que adiciona
uma camada a mais de complexidade.
\par Já existem inteligências artificiais feitas para jogar
\textbf{Magic}, mas devido à complexidade do jogo, nenhuma inteligência
artificial consegue ser realmente boa no jogo (por exemplo, elas não
costumam levar em consideração as cartas que estão na sua mão, então é
muito fácil fazer com que a IA sempre ``morda sua isca'').
\par Nosso objetivo com o trabalho será entender o problema e criar a
nossa versão de uma inteligência artificial para uma versão simplificada
do jogo (limitando o conjunto de cartas disponíveis, adicionando
restrições à construção dos baralhos e simplificando algumas regras do
jogo).

\section{Objetivos}

\begin{itemize}
    \item{} Modelar com diferentes níveis de complexidade as decisões
possíveis e representações do espaço de estados.
    \item{} Implementar diferentes algoritmos de Inteligência Artificial
em um cliente-jogador e comparar os resultados.
    \item{} Implementação e treino de uma ou mais IAs baseadas em
aprendizado de máquina.
    \item{} Usar jogos entre duas IAs (próprias ou existentes) para
avaliação de desempenho.
\end{itemize}

\section{Regras básicas}

Um jogo usual de \textit{Magic: the Gathering} conta com dois jogadores
munidos de um baralho de 60 \textit{cards}, ambos começando com 20
pontos de vida, com o objetivo de reduzir o total de pontos de vida do
oponente a 0. Para tanto, é preciso usar os \textit{cards} disponíveis
na mão, que podem representar feitiços, criaturas ou terrenos (existem
outros tipos de \textit{cards}, mas para nossa implementação vamos focar
nesses três). Feitiços são \textit{cards} que têm um efeito que acontece
no momento em que são jogados e então são colocados no cemitério (como é
chamada a pilha de descarte no jogo).\\

\begin{wrapfigure}{R}{5cm}
    \centering
    \includegraphics[width=0.25\textwidth]{picstcc/divination.jpg}
    \caption{Divinação (Feitiço)}
    \label{divination}
\end{wrapfigure}

Por exemplo, o feitiço Divinação tem um efeito simples: "Compre dois
cards". O jogador que joga este card pega os dois cards do topo de seu
baralho e os coloca em sua mão, aumentando o leque de possibilidades.
Assim, \textbf{feitiços} podem alterar o estado do jogo de diversas
maneiras (como fazer com que os jogadores comprem ou descartem cards,
alterar o total de pontos de vida de um jogador ou destruir uma
criatura) e são a principal forma de interagir com o oponente ou
desenvolver o seu lado do campo de batalha (como é chamada a zona em que
ficam as criaturas e terrenos).

\textbf{Criaturas} são cards permanentes (uma vez jogados eles
permanecem no campo de batalha até que sejam destruídos por algum
feitiço ou durante o combate) que possuem poder (quantidade de dano
causado em combate), resistência (quantidade de dano necessária para ser
destruída) e muitas vezes habilidades que afetam o andamento do combate
ou que fazem algum efeito no momento em que são jogadas.

\begin{wrapfigure}{L}{5cm}
\centering
    \includegraphics[width=0.25\textwidth]{picstcc/angelOfMercy.jpg}
    \caption{Anjo da Misericórdia (Criatura)}
    \label{anjo}
\end{wrapfigure}

Por exemplo, Anjo da Misericórdia tem a habilidade de Voar (que limita
as interações do oponente durante o combate) e concede a seu controlador
3 pontos de vida ao entrar no campo de batalha. Além disso, seu poder e
resistência são 3 e 3, respectivamente, conforme indicado na caixa no
canto inferior direito.

\textbf{Terrenos} são a fonte de \textit{mana}, que é o recurso
utilizado para pagar por criaturas e feitiços. O custo de \textit{mana}
dos \textit{cards} que não são terreno está indicado no canto superior
direito do card (por exemplo, para jogar Divinação é necessário usar
terrenos, sendo que um deles deve ser necessariamente uma Ilha, como na
\ref{ilha}). Terrenos são, portanto, um dos tipos de \textit{cards}
mais importantes, pois sem eles não há maneira (normalmente) de jogar
seus outros \textit{cards}. A mecânica associada ao uso de recursos é
\textbf{virar} (em inglês, \textit{tap}): para conjurar

\begin{wrapfigure}{R}{5cm}
    \centering
    \includegraphics[width=0.25\textwidth]{picstcc/island.jpg}
    \label{ilha}
    \caption{Ilha (Terreno)}
\end{wrapfigure}

Cada ``cor'' de \textit{mana} tem um tipo de terreno associado: assim
como azul (Ilha), temos o branco (Planície), verde (Floresta),
vermelho (Montanha) e preto (Pântano). O jogo inclusive conta com uma
espécie de ``diagrama filosófico'' chamado \textit{Color Pie}
(do inglês, ``torta das cores'', pois se trata de um gráfico circular)
associando cada cor a valores e traços de personalidade.
A título de curiosidade, alguns dos principais aspectos de cada cor são:
conhecimento, para o azul; ordem, em branco; instinto, no verde;
impulso, para o vermelho e amoralidade, característica do preto. Os
aspectos de cada cor são refletidos mecanicamente nas cartas e,
consequentemente, acabam muitas vezes refletindo, em nível mais
abstrato, nas estratégias associadas à montagem e uso dos
\textit{decks}.
Por exemplo: comprar \textit{cards} representa acumular conhecimento,
portanto Divinação é um feitiço que representa bem sua cor.
Entraremos mais a fundo no assunto na seção (INSERIR SEÇÂO), na
apresentação dos decks usados para teste e jogo.

No começo do jogo é decidido aleatoriamente quem será o jogador inicial,
e então os dois jogadores compram uma mão inicial de sete cartas.
Antes do jogo propriamente dito começar, os jogadores tomam as decisões
de \textit{mulligan}, um processo onde cada jogador pode decidir
se aceita a sua mão inicial ou se quer embaralhar de volta suas cartas
no deck e comprar uma nova mão com uma carta a menos, repetir o processo
até que decida manter a mão. Uma vez que os dois jogadores tiverem
escolhido uma mão inicial para manter, cada jogador que realizou pelo
menos um mulligan
olha a carta do topo de seu deck e decide se quer colocá-la no fundo.

O jogo então começa, com os jogadores alternando entre turnos, onde o
jogador que ``controla o turno'' é chamado de \textit{jogador ativo},
com a seguinte estrutura, simplificada (iremos entrar em maiores
detalhes nas próximas seções):

\begin{itemize}
    \item\textbf{Início do turno}: Permanentes do jogador ativo são
desviradas. Jogador ativo compra uma carta de seu \textit{deck}.
    \item\textbf{Primeira Fase Principal}: Jogador ativo pode jogar as
cartas da mão.
    \item \textbf{Combate}: Jogador ativo ``declara atacantes'' (escolhe
quais de suas criaturas irão atacar seu oponente); em seguida, seu
oponente ``declara bloqueadores'' (escolhe quais de suas criaturas irão
bloquear as criaturas atacantes). Cada criatura não-bloqueada, então,
causa dano igual ao seu poder ao oponente e todas as criaturas
bloqueadas e bloqueadoras causam dano entre si.
    \item \textbf{Segunda Fase Principal}: Igual à primeira Fase
Principal.
\end{itemize}

A estrutura acima se repete até o jogo terminar, o que acontece
geralmente quando algum jogador chega a 0 pontos de vida,
mas também pode acontecer de outras maneiras como, por exemplo, se o
baralho de um jogador acabar.

\chapter{Modelagem}

Neste capítulo abordaremos um jogo de \textit{Magic} como um problema de
Inteligência Artificial. A estrutura de um turno, como descrita
 no capítulo anterior, será repetida algumas vezes até o jogo acabar,
porém é necessário determinar a mão inicial de cada jogador e, por isso,
 trataremos este problema em separado.

\section{\textit{Mulligan}}
Baseando-se na experiência própria, a estrutura do problema do
\textbf{\textit{mulligan}} é notavelmente diferente do resto de um jogo
de \textit{Magic}.
A principal diferença é que apesar de ambos os jogadores tomarem
decisões alternadamente nessa etapa, as ações do oponente não têm
nenhuma influência
direta sobre as ações do agente, que deve se concentrar em obter uma
\textbf{mão inicial} que possibilite jogadas nos primeiros turnos do
jogo.
A figura \ref{mulligan} representa as escolhas que o jogador poderá
fazer para determiná-la, com cada conjunto de cartas representando um
conjunto
de estados.

\begin{figure}
  \centering
  \label{mulligan}
  % Graphic for TeX using PGF
% Title: /home/schutzer/comp/mtcc_super/mtcc_support/mulligan.dia
% Creator: Dia v0.97.3
% CreationDate: Fri Nov 10 11:39:18 2017
% For: schutzer
% \usepackage{tikz}
% The following commands are not supported in PSTricks at present
% We define them conditionally, so when they are implemented,
% this pgf file will use them.
\ifx\du\undefined
  \newlength{\du}
\fi
\setlength{\du}{15\unitlength}
\begin{tikzpicture}[scale=0.35]
\pgftransformxscale{1.000000}
\pgftransformyscale{-1.000000}
\definecolor{dialinecolor}{rgb}{0.000000, 0.000000, 0.000000}
\pgfsetstrokecolor{dialinecolor}
\definecolor{dialinecolor}{rgb}{1.000000, 1.000000, 1.000000}
\pgfsetfillcolor{dialinecolor}
\pgfsetlinewidth{0.100000\du}
\pgfsetdash{}{0pt}
\pgfsetdash{}{0pt}
\pgfsetmiterjoin
\definecolor{dialinecolor}{rgb}{1.000000, 1.000000, 1.000000}
\pgfsetfillcolor{dialinecolor}
\fill (31.000000\du,12.231300\du)--(31.000000\du,19.231300\du)--(36.000000\du,19.231300\du)--(36.000000\du,12.231300\du)--cycle;
\definecolor{dialinecolor}{rgb}{0.000000, 0.000000, 0.000000}
\pgfsetstrokecolor{dialinecolor}
\draw (31.000000\du,12.231300\du)--(31.000000\du,19.231300\du)--(36.000000\du,19.231300\du)--(36.000000\du,12.231300\du)--cycle;
\pgfsetlinewidth{0.100000\du}
\pgfsetdash{}{0pt}
\pgfsetdash{}{0pt}
\pgfsetmiterjoin
\definecolor{dialinecolor}{rgb}{1.000000, 1.000000, 1.000000}
\pgfsetfillcolor{dialinecolor}
\fill (30.000000\du,11.231300\du)--(30.000000\du,18.231300\du)--(35.000000\du,18.231300\du)--(35.000000\du,11.231300\du)--cycle;
\definecolor{dialinecolor}{rgb}{0.000000, 0.000000, 0.000000}
\pgfsetstrokecolor{dialinecolor}
\draw (30.000000\du,11.231300\du)--(30.000000\du,18.231300\du)--(35.000000\du,18.231300\du)--(35.000000\du,11.231300\du)--cycle;
\pgfsetlinewidth{0.100000\du}
\pgfsetdash{}{0pt}
\pgfsetdash{}{0pt}
\pgfsetmiterjoin
\definecolor{dialinecolor}{rgb}{1.000000, 1.000000, 1.000000}
\pgfsetfillcolor{dialinecolor}
\fill (29.000000\du,10.231300\du)--(29.000000\du,17.231300\du)--(34.000000\du,17.231300\du)--(34.000000\du,10.231300\du)--cycle;
\definecolor{dialinecolor}{rgb}{0.000000, 0.000000, 0.000000}
\pgfsetstrokecolor{dialinecolor}
\draw (29.000000\du,10.231300\du)--(29.000000\du,17.231300\du)--(34.000000\du,17.231300\du)--(34.000000\du,10.231300\du)--cycle;
\pgfsetlinewidth{0.100000\du}
\pgfsetdash{}{0pt}
\pgfsetdash{}{0pt}
\pgfsetmiterjoin
\definecolor{dialinecolor}{rgb}{1.000000, 1.000000, 1.000000}
\pgfsetfillcolor{dialinecolor}
\fill (28.000000\du,9.231330\du)--(28.000000\du,16.231330\du)--(33.000000\du,16.231330\du)--(33.000000\du,9.231330\du)--cycle;
\definecolor{dialinecolor}{rgb}{0.000000, 0.000000, 0.000000}
\pgfsetstrokecolor{dialinecolor}
\draw (28.000000\du,9.231330\du)--(28.000000\du,16.231330\du)--(33.000000\du,16.231330\du)--(33.000000\du,9.231330\du)--cycle;
\pgfsetlinewidth{0.100000\du}
\pgfsetdash{}{0pt}
\pgfsetdash{}{0pt}
\pgfsetmiterjoin
\definecolor{dialinecolor}{rgb}{1.000000, 1.000000, 1.000000}
\pgfsetfillcolor{dialinecolor}
\fill (27.000000\du,8.231330\du)--(27.000000\du,15.231330\du)--(32.000000\du,15.231330\du)--(32.000000\du,8.231330\du)--cycle;
\definecolor{dialinecolor}{rgb}{0.000000, 0.000000, 0.000000}
\pgfsetstrokecolor{dialinecolor}
\draw (27.000000\du,8.231330\du)--(27.000000\du,15.231330\du)--(32.000000\du,15.231330\du)--(32.000000\du,8.231330\du)--cycle;
\pgfsetlinewidth{0.100000\du}
\pgfsetdash{}{0pt}
\pgfsetdash{}{0pt}
\pgfsetmiterjoin
\definecolor{dialinecolor}{rgb}{1.000000, 1.000000, 1.000000}
\pgfsetfillcolor{dialinecolor}
\fill (26.000000\du,7.231330\du)--(26.000000\du,14.231330\du)--(31.000000\du,14.231330\du)--(31.000000\du,7.231330\du)--cycle;
\definecolor{dialinecolor}{rgb}{0.000000, 0.000000, 0.000000}
\pgfsetstrokecolor{dialinecolor}
\draw (26.000000\du,7.231330\du)--(26.000000\du,14.231330\du)--(31.000000\du,14.231330\du)--(31.000000\du,7.231330\du)--cycle;
\pgfsetlinewidth{0.100000\du}
\pgfsetdash{}{0pt}
\pgfsetdash{}{0pt}
\pgfsetmiterjoin
\definecolor{dialinecolor}{rgb}{1.000000, 1.000000, 1.000000}
\pgfsetfillcolor{dialinecolor}
\fill (25.000000\du,6.231330\du)--(25.000000\du,13.231330\du)--(30.000000\du,13.231330\du)--(30.000000\du,6.231330\du)--cycle;
\definecolor{dialinecolor}{rgb}{0.000000, 0.000000, 0.000000}
\pgfsetstrokecolor{dialinecolor}
\draw (25.000000\du,6.231330\du)--(25.000000\du,13.231330\du)--(30.000000\du,13.231330\du)--(30.000000\du,6.231330\du)--cycle;
\pgfsetlinewidth{0.100000\du}
\pgfsetdash{}{0pt}
\pgfsetdash{}{0pt}
\pgfsetmiterjoin
\definecolor{dialinecolor}{rgb}{1.000000, 1.000000, 1.000000}
\pgfsetfillcolor{dialinecolor}
\fill (36.000000\du,29.000000\du)--(36.000000\du,36.000000\du)--(41.000000\du,36.000000\du)--(41.000000\du,29.000000\du)--cycle;
\definecolor{dialinecolor}{rgb}{0.000000, 0.000000, 0.000000}
\pgfsetstrokecolor{dialinecolor}
\draw (36.000000\du,29.000000\du)--(36.000000\du,36.000000\du)--(41.000000\du,36.000000\du)--(41.000000\du,29.000000\du)--cycle;
\pgfsetlinewidth{0.100000\du}
\pgfsetdash{}{0pt}
\pgfsetdash{}{0pt}
\pgfsetmiterjoin
\definecolor{dialinecolor}{rgb}{1.000000, 1.000000, 1.000000}
\pgfsetfillcolor{dialinecolor}
\fill (35.000000\du,28.000000\du)--(35.000000\du,35.000000\du)--(40.000000\du,35.000000\du)--(40.000000\du,28.000000\du)--cycle;
\definecolor{dialinecolor}{rgb}{0.000000, 0.000000, 0.000000}
\pgfsetstrokecolor{dialinecolor}
\draw (35.000000\du,28.000000\du)--(35.000000\du,35.000000\du)--(40.000000\du,35.000000\du)--(40.000000\du,28.000000\du)--cycle;
\pgfsetlinewidth{0.100000\du}
\pgfsetdash{}{0pt}
\pgfsetdash{}{0pt}
\pgfsetmiterjoin
\definecolor{dialinecolor}{rgb}{1.000000, 1.000000, 1.000000}
\pgfsetfillcolor{dialinecolor}
\fill (34.000000\du,27.000000\du)--(34.000000\du,34.000000\du)--(39.000000\du,34.000000\du)--(39.000000\du,27.000000\du)--cycle;
\definecolor{dialinecolor}{rgb}{0.000000, 0.000000, 0.000000}
\pgfsetstrokecolor{dialinecolor}
\draw (34.000000\du,27.000000\du)--(34.000000\du,34.000000\du)--(39.000000\du,34.000000\du)--(39.000000\du,27.000000\du)--cycle;
\pgfsetlinewidth{0.100000\du}
\pgfsetdash{}{0pt}
\pgfsetdash{}{0pt}
\pgfsetmiterjoin
\definecolor{dialinecolor}{rgb}{1.000000, 1.000000, 1.000000}
\pgfsetfillcolor{dialinecolor}
\fill (33.000000\du,26.000000\du)--(33.000000\du,33.000000\du)--(38.000000\du,33.000000\du)--(38.000000\du,26.000000\du)--cycle;
\definecolor{dialinecolor}{rgb}{0.000000, 0.000000, 0.000000}
\pgfsetstrokecolor{dialinecolor}
\draw (33.000000\du,26.000000\du)--(33.000000\du,33.000000\du)--(38.000000\du,33.000000\du)--(38.000000\du,26.000000\du)--cycle;
\pgfsetlinewidth{0.100000\du}
\pgfsetdash{}{0pt}
\pgfsetdash{}{0pt}
\pgfsetmiterjoin
\definecolor{dialinecolor}{rgb}{1.000000, 1.000000, 1.000000}
\pgfsetfillcolor{dialinecolor}
\fill (32.000000\du,25.000000\du)--(32.000000\du,32.000000\du)--(37.000000\du,32.000000\du)--(37.000000\du,25.000000\du)--cycle;
\definecolor{dialinecolor}{rgb}{0.000000, 0.000000, 0.000000}
\pgfsetstrokecolor{dialinecolor}
\draw (32.000000\du,25.000000\du)--(32.000000\du,32.000000\du)--(37.000000\du,32.000000\du)--(37.000000\du,25.000000\du)--cycle;
\pgfsetlinewidth{0.100000\du}
\pgfsetdash{}{0pt}
\pgfsetdash{}{0pt}
\pgfsetmiterjoin
\definecolor{dialinecolor}{rgb}{1.000000, 1.000000, 1.000000}
\pgfsetfillcolor{dialinecolor}
\fill (31.000000\du,24.000000\du)--(31.000000\du,31.000000\du)--(36.000000\du,31.000000\du)--(36.000000\du,24.000000\du)--cycle;
\definecolor{dialinecolor}{rgb}{0.000000, 0.000000, 0.000000}
\pgfsetstrokecolor{dialinecolor}
\draw (31.000000\du,24.000000\du)--(31.000000\du,31.000000\du)--(36.000000\du,31.000000\du)--(36.000000\du,24.000000\du)--cycle;
\definecolor{dialinecolor}{rgb}{1.000000, 1.000000, 1.000000}
\pgfsetfillcolor{dialinecolor}
\fill (54.005330\du,14.000000\du)--(56.010660\du,15.752665\du)--(54.005330\du,17.505330\du)--(52.000000\du,15.752665\du)--cycle;
\pgfsetlinewidth{0.100000\du}
\pgfsetdash{}{0pt}
\pgfsetdash{}{0pt}
\pgfsetmiterjoin
\definecolor{dialinecolor}{rgb}{0.000000, 0.000000, 0.000000}
\pgfsetstrokecolor{dialinecolor}
\draw (54.005330\du,14.000000\du)--(56.010660\du,15.752665\du)--(54.005330\du,17.505330\du)--(52.000000\du,15.752665\du)--cycle;
% setfont left to latex
\definecolor{dialinecolor}{rgb}{0.000000, 0.000000, 0.000000}
\pgfsetstrokecolor{dialinecolor}
\node at (54.005330\du,15.947665\du){};
\pgfsetlinewidth{0.100000\du}
\pgfsetdash{}{0pt}
\pgfsetdash{}{0pt}
\pgfsetbuttcap
{
\definecolor{dialinecolor}{rgb}{0.000000, 0.000000, 0.000000}
\pgfsetfillcolor{dialinecolor}
% was here!!!
\pgfsetarrowsend{stealth}
\definecolor{dialinecolor}{rgb}{0.000000, 0.000000, 0.000000}
\pgfsetstrokecolor{dialinecolor}
\draw (36.049399\du,15.733956\du)--(51.950000\du,15.750524\du);
}
\definecolor{dialinecolor}{rgb}{1.000000, 1.000000, 1.000000}
\pgfsetfillcolor{dialinecolor}
\pgfpathellipse{\pgfpoint{70.343600\du}{15.752300\du}}{\pgfpoint{2.500000\du}{0\du}}{\pgfpoint{0\du}{2.500000\du}}
\pgfusepath{fill}
\pgfsetlinewidth{0.100000\du}
\pgfsetdash{}{0pt}
\pgfsetdash{}{0pt}
\pgfsetmiterjoin
\definecolor{dialinecolor}{rgb}{0.000000, 0.000000, 0.000000}
\pgfsetstrokecolor{dialinecolor}
\pgfpathellipse{\pgfpoint{70.343600\du}{15.752300\du}}{\pgfpoint{2.500000\du}{0\du}}{\pgfpoint{0\du}{2.500000\du}}
\pgfusepath{stroke}
% setfont left to latex
\definecolor{dialinecolor}{rgb}{0.000000, 0.000000, 0.000000}
\pgfsetstrokecolor{dialinecolor}
\node at (70.343600\du,15.947300\du){};
\definecolor{dialinecolor}{rgb}{1.000000, 1.000000, 1.000000}
\pgfsetfillcolor{dialinecolor}
\fill (54.005330\du,30.691600\du)--(56.010660\du,32.444265\du)--(54.005330\du,34.196930\du)--(52.000000\du,32.444265\du)--cycle;
\pgfsetlinewidth{0.100000\du}
\pgfsetdash{}{0pt}
\pgfsetdash{}{0pt}
\pgfsetmiterjoin
\definecolor{dialinecolor}{rgb}{0.000000, 0.000000, 0.000000}
\pgfsetstrokecolor{dialinecolor}
\draw (54.005330\du,30.691600\du)--(56.010660\du,32.444265\du)--(54.005330\du,34.196930\du)--(52.000000\du,32.444265\du)--cycle;
% setfont left to latex
\definecolor{dialinecolor}{rgb}{0.000000, 0.000000, 0.000000}
\pgfsetstrokecolor{dialinecolor}
\node at (54.005330\du,32.639265\du){};
\pgfsetlinewidth{0.100000\du}
\pgfsetdash{}{0pt}
\pgfsetdash{}{0pt}
\pgfsetmiterjoin
\definecolor{dialinecolor}{rgb}{1.000000, 1.000000, 1.000000}
\pgfsetfillcolor{dialinecolor}
\fill (40.000000\du,44.000000\du)--(40.000000\du,51.000000\du)--(45.000000\du,51.000000\du)--(45.000000\du,44.000000\du)--cycle;
\definecolor{dialinecolor}{rgb}{0.000000, 0.000000, 0.000000}
\pgfsetstrokecolor{dialinecolor}
\draw (40.000000\du,44.000000\du)--(40.000000\du,51.000000\du)--(45.000000\du,51.000000\du)--(45.000000\du,44.000000\du)--cycle;
\pgfsetlinewidth{0.100000\du}
\pgfsetdash{}{0pt}
\pgfsetdash{}{0pt}
\pgfsetmiterjoin
\definecolor{dialinecolor}{rgb}{1.000000, 1.000000, 1.000000}
\pgfsetfillcolor{dialinecolor}
\fill (39.000000\du,43.000000\du)--(39.000000\du,50.000000\du)--(44.000000\du,50.000000\du)--(44.000000\du,43.000000\du)--cycle;
\definecolor{dialinecolor}{rgb}{0.000000, 0.000000, 0.000000}
\pgfsetstrokecolor{dialinecolor}
\draw (39.000000\du,43.000000\du)--(39.000000\du,50.000000\du)--(44.000000\du,50.000000\du)--(44.000000\du,43.000000\du)--cycle;
\pgfsetlinewidth{0.100000\du}
\pgfsetdash{}{0pt}
\pgfsetdash{}{0pt}
\pgfsetmiterjoin
\definecolor{dialinecolor}{rgb}{1.000000, 1.000000, 1.000000}
\pgfsetfillcolor{dialinecolor}
\fill (38.000000\du,42.000000\du)--(38.000000\du,49.000000\du)--(43.000000\du,49.000000\du)--(43.000000\du,42.000000\du)--cycle;
\definecolor{dialinecolor}{rgb}{0.000000, 0.000000, 0.000000}
\pgfsetstrokecolor{dialinecolor}
\draw (38.000000\du,42.000000\du)--(38.000000\du,49.000000\du)--(43.000000\du,49.000000\du)--(43.000000\du,42.000000\du)--cycle;
\pgfsetlinewidth{0.100000\du}
\pgfsetdash{}{0pt}
\pgfsetdash{}{0pt}
\pgfsetmiterjoin
\definecolor{dialinecolor}{rgb}{1.000000, 1.000000, 1.000000}
\pgfsetfillcolor{dialinecolor}
\fill (37.000000\du,41.000000\du)--(37.000000\du,48.000000\du)--(42.000000\du,48.000000\du)--(42.000000\du,41.000000\du)--cycle;
\definecolor{dialinecolor}{rgb}{0.000000, 0.000000, 0.000000}
\pgfsetstrokecolor{dialinecolor}
\draw (37.000000\du,41.000000\du)--(37.000000\du,48.000000\du)--(42.000000\du,48.000000\du)--(42.000000\du,41.000000\du)--cycle;
\pgfsetlinewidth{0.100000\du}
\pgfsetdash{}{0pt}
\pgfsetdash{}{0pt}
\pgfsetmiterjoin
\definecolor{dialinecolor}{rgb}{1.000000, 1.000000, 1.000000}
\pgfsetfillcolor{dialinecolor}
\fill (36.000000\du,40.000000\du)--(36.000000\du,47.000000\du)--(41.000000\du,47.000000\du)--(41.000000\du,40.000000\du)--cycle;
\definecolor{dialinecolor}{rgb}{0.000000, 0.000000, 0.000000}
\pgfsetstrokecolor{dialinecolor}
\draw (36.000000\du,40.000000\du)--(36.000000\du,47.000000\du)--(41.000000\du,47.000000\du)--(41.000000\du,40.000000\du)--cycle;
\pgfsetlinewidth{0.100000\du}
\pgfsetdash{}{0pt}
\pgfsetdash{}{0pt}
\pgfsetmiterjoin
\definecolor{dialinecolor}{rgb}{1.000000, 1.000000, 1.000000}
\pgfsetfillcolor{dialinecolor}
\fill (40.000000\du,68.000000\du)--(40.000000\du,75.000000\du)--(45.000000\du,75.000000\du)--(45.000000\du,68.000000\du)--cycle;
\definecolor{dialinecolor}{rgb}{0.000000, 0.000000, 0.000000}
\pgfsetstrokecolor{dialinecolor}
\draw (40.000000\du,68.000000\du)--(40.000000\du,75.000000\du)--(45.000000\du,75.000000\du)--(45.000000\du,68.000000\du)--cycle;
% setfont left to latex
\definecolor{dialinecolor}{rgb}{0.000000, 0.000000, 0.000000}
\pgfsetstrokecolor{dialinecolor}
\node[anchor=west] at (59.500000\du,14.300000\du){$K$};
\pgfsetlinewidth{0.100000\du}
\pgfsetdash{}{0pt}
\pgfsetdash{}{0pt}
\pgfsetbuttcap
{
\definecolor{dialinecolor}{rgb}{0.000000, 0.000000, 0.000000}
\pgfsetfillcolor{dialinecolor}
% was here!!!
\pgfsetarrowsend{stealth}
\definecolor{dialinecolor}{rgb}{0.000000, 0.000000, 0.000000}
\pgfsetstrokecolor{dialinecolor}
\draw (41.049789\du,32.489480\du)--(52.000000\du,32.444300\du);
}
\definecolor{dialinecolor}{rgb}{1.000000, 1.000000, 1.000000}
\pgfsetfillcolor{dialinecolor}
\fill (54.005330\du,45.793300\du)--(56.010660\du,47.545965\du)--(54.005330\du,49.298630\du)--(52.000000\du,47.545965\du)--cycle;
\pgfsetlinewidth{0.100000\du}
\pgfsetdash{}{0pt}
\pgfsetdash{}{0pt}
\pgfsetmiterjoin
\definecolor{dialinecolor}{rgb}{0.000000, 0.000000, 0.000000}
\pgfsetstrokecolor{dialinecolor}
\draw (54.005330\du,45.793300\du)--(56.010660\du,47.545965\du)--(54.005330\du,49.298630\du)--(52.000000\du,47.545965\du)--cycle;
% setfont left to latex
\definecolor{dialinecolor}{rgb}{0.000000, 0.000000, 0.000000}
\pgfsetstrokecolor{dialinecolor}
\node at (54.005330\du,47.740965\du){};
% setfont left to latex
\definecolor{dialinecolor}{rgb}{0.000000, 0.000000, 0.000000}
\pgfsetstrokecolor{dialinecolor}
\node[anchor=west] at (69.000000\du,68.116600\du){};
% setfont left to latex
\definecolor{dialinecolor}{rgb}{0.000000, 0.000000, 0.000000}
\pgfsetstrokecolor{dialinecolor}
\node[anchor=west] at (69.000000\du,46.294500\du){};
\definecolor{dialinecolor}{rgb}{1.000000, 1.000000, 1.000000}
\pgfsetfillcolor{dialinecolor}
\fill (70.174430\du,44.500000\du)--(73.348860\du,47.603350\du)--(70.174430\du,50.706700\du)--(67.000000\du,47.603350\du)--cycle;
\pgfsetlinewidth{0.100000\du}
\pgfsetdash{}{0pt}
\pgfsetdash{}{0pt}
\pgfsetmiterjoin
\definecolor{dialinecolor}{rgb}{0.000000, 0.000000, 0.000000}
\pgfsetstrokecolor{dialinecolor}
\draw (70.174430\du,44.500000\du)--(73.348860\du,47.603350\du)--(70.174430\du,50.706700\du)--(67.000000\du,47.603350\du)--cycle;
% setfont left to latex
\definecolor{dialinecolor}{rgb}{0.000000, 0.000000, 0.000000}
\pgfsetstrokecolor{dialinecolor}
\node at (70.174430\du,47.798350\du){$scry$};
\definecolor{dialinecolor}{rgb}{1.000000, 1.000000, 1.000000}
\pgfsetfillcolor{dialinecolor}
\pgfpathellipse{\pgfpoint{65.558300\du}{56.535900\du}}{\pgfpoint{2.500000\du}{0\du}}{\pgfpoint{0\du}{2.500000\du}}
\pgfusepath{fill}
\pgfsetlinewidth{0.100000\du}
\pgfsetdash{}{0pt}
\pgfsetdash{}{0pt}
\pgfsetmiterjoin
\definecolor{dialinecolor}{rgb}{0.000000, 0.000000, 0.000000}
\pgfsetstrokecolor{dialinecolor}
\pgfpathellipse{\pgfpoint{65.558300\du}{56.535900\du}}{\pgfpoint{2.500000\du}{0\du}}{\pgfpoint{0\du}{2.500000\du}}
\pgfusepath{stroke}
% setfont left to latex
\definecolor{dialinecolor}{rgb}{0.000000, 0.000000, 0.000000}
\pgfsetstrokecolor{dialinecolor}
\node at (65.558300\du,56.730900\du){};
\definecolor{dialinecolor}{rgb}{1.000000, 1.000000, 1.000000}
\pgfsetfillcolor{dialinecolor}
\pgfpathellipse{\pgfpoint{75.549200\du}{56.535900\du}}{\pgfpoint{2.500000\du}{0\du}}{\pgfpoint{0\du}{2.500000\du}}
\pgfusepath{fill}
\pgfsetlinewidth{0.100000\du}
\pgfsetdash{}{0pt}
\pgfsetdash{}{0pt}
\pgfsetmiterjoin
\definecolor{dialinecolor}{rgb}{0.000000, 0.000000, 0.000000}
\pgfsetstrokecolor{dialinecolor}
\pgfpathellipse{\pgfpoint{75.549200\du}{56.535900\du}}{\pgfpoint{2.500000\du}{0\du}}{\pgfpoint{0\du}{2.500000\du}}
\pgfusepath{stroke}
% setfont left to latex
\definecolor{dialinecolor}{rgb}{0.000000, 0.000000, 0.000000}
\pgfsetstrokecolor{dialinecolor}
\node at (75.549200\du,56.730900\du){};
\pgfsetlinewidth{0.100000\du}
\pgfsetdash{}{0pt}
\pgfsetdash{}{0pt}
\pgfsetbuttcap
{
\definecolor{dialinecolor}{rgb}{0.000000, 0.000000, 0.000000}
\pgfsetfillcolor{dialinecolor}
% was here!!!
\pgfsetarrowsend{stealth}
\definecolor{dialinecolor}{rgb}{0.000000, 0.000000, 0.000000}
\pgfsetstrokecolor{dialinecolor}
\draw (69.091820\du,49.698279\du)--(66.728674\du,54.271141\du);
}
\pgfsetlinewidth{0.100000\du}
\pgfsetdash{}{0pt}
\pgfsetdash{}{0pt}
\pgfsetbuttcap
{
\definecolor{dialinecolor}{rgb}{0.000000, 0.000000, 0.000000}
\pgfsetfillcolor{dialinecolor}
% was here!!!
\pgfsetarrowsend{stealth}
\definecolor{dialinecolor}{rgb}{0.000000, 0.000000, 0.000000}
\pgfsetstrokecolor{dialinecolor}
\draw (71.368860\du,49.588422\du)--(74.235688\du,54.352921\du);
}
\pgfsetlinewidth{0.100000\du}
\pgfsetdash{}{0pt}
\pgfsetdash{}{0pt}
\pgfsetbuttcap
{
\definecolor{dialinecolor}{rgb}{0.000000, 0.000000, 0.000000}
\pgfsetfillcolor{dialinecolor}
% was here!!!
\pgfsetarrowsend{stealth}
\definecolor{dialinecolor}{rgb}{0.000000, 0.000000, 0.000000}
\pgfsetstrokecolor{dialinecolor}
\draw (56.056369\du,47.555023\du)--(67.000000\du,47.603350\du);
}
\pgfsetlinewidth{0.100000\du}
\pgfsetdash{}{0pt}
\pgfsetdash{}{0pt}
\pgfsetbuttcap
{
\definecolor{dialinecolor}{rgb}{0.000000, 0.000000, 0.000000}
\pgfsetfillcolor{dialinecolor}
% was here!!!
\pgfsetarrowsend{stealth}
\definecolor{dialinecolor}{rgb}{0.000000, 0.000000, 0.000000}
\pgfsetstrokecolor{dialinecolor}
\draw (45.000000\du,47.500000\du)--(52.000000\du,47.546000\du);
}
\definecolor{dialinecolor}{rgb}{1.000000, 1.000000, 1.000000}
\pgfsetfillcolor{dialinecolor}
\fill (54.005330\du,69.762200\du)--(56.010660\du,71.514865\du)--(54.005330\du,73.267530\du)--(52.000000\du,71.514865\du)--cycle;
\pgfsetlinewidth{0.100000\du}
\pgfsetdash{}{0pt}
\pgfsetdash{}{0pt}
\pgfsetmiterjoin
\definecolor{dialinecolor}{rgb}{0.000000, 0.000000, 0.000000}
\pgfsetstrokecolor{dialinecolor}
\draw (54.005330\du,69.762200\du)--(56.010660\du,71.514865\du)--(54.005330\du,73.267530\du)--(52.000000\du,71.514865\du)--cycle;
% setfont left to latex
\definecolor{dialinecolor}{rgb}{0.000000, 0.000000, 0.000000}
\pgfsetstrokecolor{dialinecolor}
\node at (54.005330\du,71.709865\du){};
\pgfsetlinewidth{0.100000\du}
\pgfsetdash{}{0pt}
\pgfsetdash{}{0pt}
\pgfsetbuttcap
{
\definecolor{dialinecolor}{rgb}{0.000000, 0.000000, 0.000000}
\pgfsetfillcolor{dialinecolor}
% was here!!!
\pgfsetarrowsend{stealth}
\definecolor{dialinecolor}{rgb}{0.000000, 0.000000, 0.000000}
\pgfsetstrokecolor{dialinecolor}
\draw (45.049530\du,71.503999\du)--(52.000000\du,71.514900\du);
}
% setfont left to latex
\definecolor{dialinecolor}{rgb}{0.000000, 0.000000, 0.000000}
\pgfsetstrokecolor{dialinecolor}
\node[anchor=east] at (18.000000\du,15.917000\du){$i=7$};
% setfont left to latex
\definecolor{dialinecolor}{rgb}{0.000000, 0.000000, 0.000000}
\pgfsetstrokecolor{dialinecolor}
\node[anchor=east] at (18.000000\du,32.275100\du){6};
% setfont left to latex
\definecolor{dialinecolor}{rgb}{0.000000, 0.000000, 0.000000}
\pgfsetstrokecolor{dialinecolor}
\node[anchor=east] at (18.000000\du,48.000000\du){5};
% setfont left to latex
\definecolor{dialinecolor}{rgb}{0.000000, 0.000000, 0.000000}
\pgfsetstrokecolor{dialinecolor}
\node[anchor=east] at (18.000000\du,72.000000\du){1};
% setfont left to latex
\definecolor{dialinecolor}{rgb}{0.000000, 0.000000, 0.000000}
\pgfsetstrokecolor{dialinecolor}
\node[anchor=east] at (18.000000\du,85.000000\du){0};

\pgfsetlinewidth{0.100000\du}
\pgfsetbuttcap
{
\definecolor{dialinecolor}{rgb}{0.000000, 0.000000, 0.000000}
\pgfsetfillcolor{dialinecolor}
% was here!!!
\pgfsetarrowsend{stealth}
\definecolor{dialinecolor}{rgb}{0.000000, 0.000000, 0.000000}
\pgfsetstrokecolor{dialinecolor}
\draw (56.060660\du,15.752611\du)--(67.843600\du,15.752300\du);
}
% setfont left to latex
\definecolor{dialinecolor}{rgb}{0.000000, 0.000000, 0.000000}
\pgfsetstrokecolor{dialinecolor}
\node[anchor=west] at (59.500000\du,31.000000\du){$K$};
% setfont left to latex
\definecolor{dialinecolor}{rgb}{0.000000, 0.000000, 0.000000}
\pgfsetstrokecolor{dialinecolor}
\node[anchor=west] at (59.500000\du,46.200000\du){$K$};
% setfont left to latex
\definecolor{dialinecolor}{rgb}{0.000000, 0.000000, 0.000000}
\pgfsetstrokecolor{dialinecolor}
\node[anchor=west] at (59.500000\du,70.000000\du){$K$};

\pgfsetlinewidth{0.100000\du}
\pgfsetdash{}{0pt}
\pgfsetdash{}{0pt}
\pgfsetbuttcap
{
\definecolor{dialinecolor}{rgb}{0.000000, 0.000000, 0.000000}
\pgfsetfillcolor{dialinecolor}
% was here!!!
\pgfsetarrowsend{stealth}
\definecolor{dialinecolor}{rgb}{0.000000, 0.000000, 0.000000}
\pgfsetstrokecolor{dialinecolor}
\draw (53.002700\du,16.629000\du)--(33.500000\du,24.000000\du);
}
\pgfsetlinewidth{0.100000\du}
\pgfsetdash{}{0pt}
\pgfsetdash{}{0pt}
\pgfsetbuttcap
{
\definecolor{dialinecolor}{rgb}{0.000000, 0.000000, 0.000000}
\pgfsetfillcolor{dialinecolor}
% was here!!!
\pgfsetarrowsend{stealth}
\definecolor{dialinecolor}{rgb}{0.000000, 0.000000, 0.000000}
\pgfsetstrokecolor{dialinecolor}
\draw (53.002700\du,33.320600\du)--(38.500000\du,40.000000\du);
}
\pgfsetlinewidth{0.100000\du}
\pgfsetdash{{1.000000\du}{1.000000\du}}{0\du}
\pgfsetdash{{1.000000\du}{1.000000\du}}{0\du}
\pgfsetbuttcap
{
\definecolor{dialinecolor}{rgb}{0.000000, 0.000000, 0.000000}
\pgfsetfillcolor{dialinecolor}
% was here!!!
\pgfsetarrowsend{stealth}
\definecolor{dialinecolor}{rgb}{0.000000, 0.000000, 0.000000}
\pgfsetstrokecolor{dialinecolor}
\draw (53.002700\du,48.422300\du)--(42.500000\du,68.000000\du);
}
\definecolor{dialinecolor}{rgb}{0.000000, 0.000000, 0.000000}
\pgfsetstrokecolor{dialinecolor}
\draw (52.529966\du,49.303505\du)--(42.500000\du,68.000000\du);
\pgfsetlinewidth{0.100000\du}
\pgfsetdash{}{0pt}
\definecolor{dialinecolor}{rgb}{0.000000, 0.000000, 0.000000}
\pgfsetstrokecolor{dialinecolor}
\draw (53.002700\du,48.422300\du)--(52.908153\du,48.598541\du);
\definecolor{dialinecolor}{rgb}{0.000000, 0.000000, 0.000000}
\pgfsetstrokecolor{dialinecolor}
\draw (52.845122\du,48.716035\du)--(52.750575\du,48.892276\du);
\definecolor{dialinecolor}{rgb}{0.000000, 0.000000, 0.000000}
\pgfsetstrokecolor{dialinecolor}
\draw (52.687544\du,49.009770\du)--(52.592998\du,49.186011\du);
% setfont left to latex
\definecolor{dialinecolor}{rgb}{0.000000, 0.000000, 0.000000}
\pgfsetstrokecolor{dialinecolor}
\node[anchor=west] at (43.000000\du,22.000000\du){$M$};
% setfont left to latex
\definecolor{dialinecolor}{rgb}{0.000000, 0.000000, 0.000000}
\pgfsetstrokecolor{dialinecolor}
\node[anchor=west] at (48.000000\du,59.000000\du){$M$};
% setfont left to latex
\definecolor{dialinecolor}{rgb}{0.000000, 0.000000, 0.000000}
\pgfsetstrokecolor{dialinecolor}
\node[anchor=west] at (46.000000\du,38.400000\du){$M$};
% setfont left to latex
\definecolor{dialinecolor}{rgb}{0.000000, 0.000000, 0.000000}
\pgfsetstrokecolor{dialinecolor}
\node[anchor=west] at (48.000000\du,78.000000\du){$M$};
% setfont left to latex
\definecolor{dialinecolor}{rgb}{0.000000, 0.000000, 0.000000}
\pgfsetstrokecolor{dialinecolor}
\node[anchor=west] at (70.174430\du,47.603350\du){};
% setfont left to latex
\definecolor{dialinecolor}{rgb}{0.000000, 0.000000, 0.000000}
\pgfsetstrokecolor{dialinecolor}
\node[anchor=west] at (70.343530\du,31.517665\du){};
\definecolor{dialinecolor}{rgb}{1.000000, 1.000000, 1.000000}
\pgfsetfillcolor{dialinecolor}
\fill (70.116130\du,29.264100\du)--(73.290560\du,32.367450\du)--(70.116130\du,35.470800\du)--(66.941700\du,32.367450\du)--cycle;
\pgfsetlinewidth{0.100000\du}
\pgfsetdash{}{0pt}
\pgfsetdash{}{0pt}
\pgfsetmiterjoin
\definecolor{dialinecolor}{rgb}{0.000000, 0.000000, 0.000000}
\pgfsetstrokecolor{dialinecolor}
\draw (70.116130\du,29.264100\du)--(73.290560\du,32.367450\du)--(70.116130\du,35.470800\du)--(66.941700\du,32.367450\du)--cycle;
% setfont left to latex
\definecolor{dialinecolor}{rgb}{0.000000, 0.000000, 0.000000}
\pgfsetstrokecolor{dialinecolor}
\node at (70.116130\du,32.562450\du){$scry$};
\definecolor{dialinecolor}{rgb}{1.000000, 1.000000, 1.000000}
\pgfsetfillcolor{dialinecolor}
\pgfpathellipse{\pgfpoint{65.500000\du}{41.300000\du}}{\pgfpoint{2.500000\du}{0\du}}{\pgfpoint{0\du}{2.500000\du}}
\pgfusepath{fill}
\pgfsetlinewidth{0.100000\du}
\pgfsetdash{}{0pt}
\pgfsetdash{}{0pt}
\pgfsetmiterjoin
\definecolor{dialinecolor}{rgb}{0.000000, 0.000000, 0.000000}
\pgfsetstrokecolor{dialinecolor}
\pgfpathellipse{\pgfpoint{65.500000\du}{41.300000\du}}{\pgfpoint{2.500000\du}{0\du}}{\pgfpoint{0\du}{2.500000\du}}
\pgfusepath{stroke}
% setfont left to latex
\definecolor{dialinecolor}{rgb}{0.000000, 0.000000, 0.000000}
\pgfsetstrokecolor{dialinecolor}
\node at (65.500000\du,41.495000\du){};
\definecolor{dialinecolor}{rgb}{1.000000, 1.000000, 1.000000}
\pgfsetfillcolor{dialinecolor}
\pgfpathellipse{\pgfpoint{75.490900\du}{41.300000\du}}{\pgfpoint{2.500000\du}{0\du}}{\pgfpoint{0\du}{2.500000\du}}
\pgfusepath{fill}
\pgfsetlinewidth{0.100000\du}
\pgfsetdash{}{0pt}
\pgfsetdash{}{0pt}
\pgfsetmiterjoin
\definecolor{dialinecolor}{rgb}{0.000000, 0.000000, 0.000000}
\pgfsetstrokecolor{dialinecolor}
\pgfpathellipse{\pgfpoint{75.490900\du}{41.300000\du}}{\pgfpoint{2.500000\du}{0\du}}{\pgfpoint{0\du}{2.500000\du}}
\pgfusepath{stroke}
% setfont left to latex
\definecolor{dialinecolor}{rgb}{0.000000, 0.000000, 0.000000}
\pgfsetstrokecolor{dialinecolor}
\node at (75.490900\du,41.495000\du){};
\pgfsetlinewidth{0.100000\du}
\pgfsetdash{}{0pt}
\pgfsetdash{}{0pt}
\pgfsetbuttcap
{
\definecolor{dialinecolor}{rgb}{0.000000, 0.000000, 0.000000}
\pgfsetfillcolor{dialinecolor}
% was here!!!
\pgfsetarrowsend{stealth}
\definecolor{dialinecolor}{rgb}{0.000000, 0.000000, 0.000000}
\pgfsetstrokecolor{dialinecolor}
\draw (69.033520\du,34.462379\du)--(66.670374\du,39.035241\du);
}
\pgfsetlinewidth{0.100000\du}
\pgfsetdash{}{0pt}
\pgfsetdash{}{0pt}
\pgfsetbuttcap
{
\definecolor{dialinecolor}{rgb}{0.000000, 0.000000, 0.000000}
\pgfsetfillcolor{dialinecolor}
% was here!!!
\pgfsetarrowsend{stealth}
\definecolor{dialinecolor}{rgb}{0.000000, 0.000000, 0.000000}
\pgfsetstrokecolor{dialinecolor}
\draw (71.310560\du,34.352522\du)--(74.177388\du,39.117021\du);
}

% setfont left to latex
\definecolor{dialinecolor}{rgb}{0.000000, 0.000000, 0.000000}
\pgfsetstrokecolor{dialinecolor}
\node[anchor=west] at (73.000000\du,37.800000\du){};
\pgfsetlinewidth{0.100000\du}
\pgfsetdash{}{0pt}
\pgfsetdash{}{0pt}
\pgfsetbuttcap
{
\definecolor{dialinecolor}{rgb}{0.000000, 0.000000, 0.000000}
\pgfsetfillcolor{dialinecolor}
% was here!!!
\pgfsetarrowsend{stealth}
\definecolor{dialinecolor}{rgb}{0.000000, 0.000000, 0.000000}
\pgfsetstrokecolor{dialinecolor}
\draw (56.053483\du,32.432103\du)--(66.941700\du,32.367450\du);
}
\definecolor{dialinecolor}{rgb}{1.000000, 1.000000, 1.000000}
\pgfsetfillcolor{dialinecolor}
\fill (70.174430\du,68.350000\du)--(73.348860\du,71.453350\du)--(70.174430\du,74.556700\du)--(67.000000\du,71.453350\du)--cycle;
\pgfsetlinewidth{0.100000\du}
\pgfsetdash{}{0pt}
\pgfsetdash{}{0pt}
\pgfsetmiterjoin
\definecolor{dialinecolor}{rgb}{0.000000, 0.000000, 0.000000}
\pgfsetstrokecolor{dialinecolor}
\draw (70.174430\du,68.350000\du)--(73.348860\du,71.453350\du)--(70.174430\du,74.556700\du)--(67.000000\du,71.453350\du)--cycle;
% setfont left to latex
\definecolor{dialinecolor}{rgb}{0.000000, 0.000000, 0.000000}
\pgfsetstrokecolor{dialinecolor}
\node at (70.174430\du,71.648350\du){$scry$};
\definecolor{dialinecolor}{rgb}{1.000000, 1.000000, 1.000000}
\pgfsetfillcolor{dialinecolor}
\pgfpathellipse{\pgfpoint{65.558300\du}{80.385900\du}}{\pgfpoint{2.500000\du}{0\du}}{\pgfpoint{0\du}{2.500000\du}}
\pgfusepath{fill}
\pgfsetlinewidth{0.100000\du}
\pgfsetdash{}{0pt}
\pgfsetdash{}{0pt}
\pgfsetmiterjoin
\definecolor{dialinecolor}{rgb}{0.000000, 0.000000, 0.000000}
\pgfsetstrokecolor{dialinecolor}
\pgfpathellipse{\pgfpoint{65.558300\du}{80.385900\du}}{\pgfpoint{2.500000\du}{0\du}}{\pgfpoint{0\du}{2.500000\du}}
\pgfusepath{stroke}
% setfont left to latex
\definecolor{dialinecolor}{rgb}{0.000000, 0.000000, 0.000000}
\pgfsetstrokecolor{dialinecolor}
\node at (65.558300\du,80.580900\du){};
\definecolor{dialinecolor}{rgb}{1.000000, 1.000000, 1.000000}
\pgfsetfillcolor{dialinecolor}
\pgfpathellipse{\pgfpoint{75.549200\du}{80.385900\du}}{\pgfpoint{2.500000\du}{0\du}}{\pgfpoint{0\du}{2.500000\du}}
\pgfusepath{fill}
\pgfsetlinewidth{0.100000\du}
\pgfsetdash{}{0pt}
\pgfsetdash{}{0pt}
\pgfsetmiterjoin
\definecolor{dialinecolor}{rgb}{0.000000, 0.000000, 0.000000}
\pgfsetstrokecolor{dialinecolor}
\pgfpathellipse{\pgfpoint{75.549200\du}{80.385900\du}}{\pgfpoint{2.500000\du}{0\du}}{\pgfpoint{0\du}{2.500000\du}}
\pgfusepath{stroke}
% setfont left to latex
\definecolor{dialinecolor}{rgb}{0.000000, 0.000000, 0.000000}
\pgfsetstrokecolor{dialinecolor}
\node at (75.549200\du,80.580900\du){};
\pgfsetlinewidth{0.100000\du}
\pgfsetdash{}{0pt}
\pgfsetdash{}{0pt}
\pgfsetbuttcap
{
\definecolor{dialinecolor}{rgb}{0.000000, 0.000000, 0.000000}
\pgfsetfillcolor{dialinecolor}
% was here!!!
\pgfsetarrowsend{stealth}
\definecolor{dialinecolor}{rgb}{0.000000, 0.000000, 0.000000}
\pgfsetstrokecolor{dialinecolor}
\draw (69.091820\du,73.548279\du)--(66.728674\du,78.121141\du);
}
\pgfsetlinewidth{0.100000\du}
\pgfsetdash{}{0pt}
\pgfsetdash{}{0pt}
\pgfsetbuttcap
{
\definecolor{dialinecolor}{rgb}{0.000000, 0.000000, 0.000000}
\pgfsetfillcolor{dialinecolor}
% was here!!!
\pgfsetarrowsend{stealth}
\definecolor{dialinecolor}{rgb}{0.000000, 0.000000, 0.000000}
\pgfsetstrokecolor{dialinecolor}
\draw (71.368860\du,73.438422\du)--(74.235688\du,78.202921\du);
}
% setfont left to latex
\definecolor{dialinecolor}{rgb}{0.000000, 0.000000, 0.000000}
\pgfsetstrokecolor{dialinecolor}
\node[anchor=west] at (37.000000\du,88.479600\du){$T$};
% setfont left to latex
\definecolor{dialinecolor}{rgb}{0.000000, 0.000000, 0.000000}
\pgfsetstrokecolor{dialinecolor}
\node[anchor=west] at (65.000000\du,35.743700\du){$T$};
% setfont left to latex
\definecolor{dialinecolor}{rgb}{0.000000, 0.000000, 0.000000}
\pgfsetstrokecolor{dialinecolor}
\node[anchor=west] at (65.000000\du,74.829600\du){$T$};
% setfont left to latex
\definecolor{dialinecolor}{rgb}{0.000000, 0.000000, 0.000000}
\pgfsetstrokecolor{dialinecolor}
\node[anchor=west] at (65.000000\du,50.979600\du){$T$};
% setfont left to latex
\definecolor{dialinecolor}{rgb}{0.000000, 0.000000, 0.000000}
\pgfsetstrokecolor{dialinecolor}
\node[anchor=west] at (72.700000\du,35.743700\du){$B$};
% setfont left to latex
\definecolor{dialinecolor}{rgb}{0.000000, 0.000000, 0.000000}
\pgfsetstrokecolor{dialinecolor}
\node[anchor=west] at (72.700000\du,50.979600\du){$B$};
% setfont left to latex
\definecolor{dialinecolor}{rgb}{0.000000, 0.000000, 0.000000}
\pgfsetstrokecolor{dialinecolor}
\node[anchor=west] at (72.700000\du,74.829600\du){$B$};
% setfont left to latex
\definecolor{dialinecolor}{rgb}{0.000000, 0.000000, 0.000000}
\pgfsetstrokecolor{dialinecolor}
\node[anchor=west] at (44.70000\du,88.479600\du){$B$};
\pgfsetlinewidth{0.100000\du}
\pgfsetdash{}{0pt}
\pgfsetdash{}{0pt}
\pgfsetbuttcap
{
\definecolor{dialinecolor}{rgb}{0.000000, 0.000000, 0.000000}
\pgfsetfillcolor{dialinecolor}
% was here!!!
\pgfsetarrowsend{stealth}
\definecolor{dialinecolor}{rgb}{0.000000, 0.000000, 0.000000}
\pgfsetstrokecolor{dialinecolor}
\draw (56.056765\du,71.505154\du)--(67.000000\du,71.453350\du);
}
\definecolor{dialinecolor}{rgb}{1.000000, 1.000000, 1.000000}
\pgfsetfillcolor{dialinecolor}
\fill (42.374430\du,82.000000\du)--(45.548860\du,85.103350\du)--(42.374430\du,88.206700\du)--(39.200000\du,85.103350\du)--cycle;
\pgfsetlinewidth{0.100000\du}
\pgfsetdash{}{0pt}
\pgfsetdash{}{0pt}
\pgfsetmiterjoin
\definecolor{dialinecolor}{rgb}{0.000000, 0.000000, 0.000000}
\pgfsetstrokecolor{dialinecolor}
\draw (42.374430\du,82.000000\du)--(45.548860\du,85.103350\du)--(42.374430\du,88.206700\du)--(39.200000\du,85.103350\du)--cycle;
% setfont left to latex
\definecolor{dialinecolor}{rgb}{0.000000, 0.000000, 0.000000}
\pgfsetstrokecolor{dialinecolor}
\node at (42.374430\du,85.298350\du){$scry$};
\definecolor{dialinecolor}{rgb}{1.000000, 1.000000, 1.000000}
\pgfsetfillcolor{dialinecolor}
\pgfpathellipse{\pgfpoint{37.758300\du}{94.035900\du}}{\pgfpoint{2.500000\du}{0\du}}{\pgfpoint{0\du}{2.500000\du}}
\pgfusepath{fill}
\pgfsetlinewidth{0.100000\du}
\pgfsetdash{}{0pt}
\pgfsetdash{}{0pt}
\pgfsetmiterjoin
\definecolor{dialinecolor}{rgb}{0.000000, 0.000000, 0.000000}
\pgfsetstrokecolor{dialinecolor}
\pgfpathellipse{\pgfpoint{37.758300\du}{94.035900\du}}{\pgfpoint{2.500000\du}{0\du}}{\pgfpoint{0\du}{2.500000\du}}
\pgfusepath{stroke}
% setfont left to latex
\definecolor{dialinecolor}{rgb}{0.000000, 0.000000, 0.000000}
\pgfsetstrokecolor{dialinecolor}
\node at (37.758300\du,94.230900\du){};
\definecolor{dialinecolor}{rgb}{1.000000, 1.000000, 1.000000}
\pgfsetfillcolor{dialinecolor}
\pgfpathellipse{\pgfpoint{47.749200\du}{94.035900\du}}{\pgfpoint{2.500000\du}{0\du}}{\pgfpoint{0\du}{2.500000\du}}
\pgfusepath{fill}
\pgfsetlinewidth{0.100000\du}
\pgfsetdash{}{0pt}
\pgfsetdash{}{0pt}
\pgfsetmiterjoin
\definecolor{dialinecolor}{rgb}{0.000000, 0.000000, 0.000000}
\pgfsetstrokecolor{dialinecolor}
\pgfpathellipse{\pgfpoint{47.749200\du}{94.035900\du}}{\pgfpoint{2.500000\du}{0\du}}{\pgfpoint{0\du}{2.500000\du}}
\pgfusepath{stroke}
% setfont left to latex
\definecolor{dialinecolor}{rgb}{0.000000, 0.000000, 0.000000}
\pgfsetstrokecolor{dialinecolor}
\node at (47.749200\du,94.230900\du){};
\pgfsetlinewidth{0.100000\du}
\pgfsetdash{}{0pt}
\pgfsetdash{}{0pt}
\pgfsetbuttcap
{
\definecolor{dialinecolor}{rgb}{0.000000, 0.000000, 0.000000}
\pgfsetfillcolor{dialinecolor}
% was here!!!
\pgfsetarrowsend{stealth}
\definecolor{dialinecolor}{rgb}{0.000000, 0.000000, 0.000000}
\pgfsetstrokecolor{dialinecolor}
\draw (41.291820\du,87.198279\du)--(38.928674\du,91.771141\du);
}
\pgfsetlinewidth{0.100000\du}
\pgfsetdash{}{0pt}
\pgfsetdash{}{0pt}
\pgfsetbuttcap
{
\definecolor{dialinecolor}{rgb}{0.000000, 0.000000, 0.000000}
\pgfsetfillcolor{dialinecolor}
% was here!!!
\pgfsetarrowsend{stealth}
\definecolor{dialinecolor}{rgb}{0.000000, 0.000000, 0.000000}
\pgfsetstrokecolor{dialinecolor}
\draw (43.568860\du,87.088422\du)--(46.435688\du,91.852921\du);
}
\pgfsetlinewidth{0.100000\du}
\pgfsetdash{}{0pt}
\pgfsetdash{}{0pt}
\pgfsetbuttcap
{
\definecolor{dialinecolor}{rgb}{0.000000, 0.000000, 0.000000}
\pgfsetfillcolor{dialinecolor}
% was here!!!
\pgfsetarrowsend{stealth}
\definecolor{dialinecolor}{rgb}{0.000000, 0.000000, 0.000000}
\pgfsetstrokecolor{dialinecolor}
\draw (53.002665\du,72.391198\du)--(42.374430\du,82.000000\du);
}
\end{tikzpicture}

  \caption{Espaço de estados de um problema de \textit{mulligan}. Cada
nível $h$ representa o número de cartas na mão inicial do jogador.}
\end{figure}

Definimos vagamente uma mão jogável se esta contém tanto cartas que
impactam o estado de jogo e recursos necessários para jogá-las.
Para este efeito, separaremos as cartas do deck em duas categorias:
terrenos (recursos) e não-terrenos (impactam o jogo).
Outro fator importante é o tamanho da mão, pois cada carta perdida
representa um turno de atraso em relação a uma mão com sete cartas (uma
vez que cada jogador compra
uma carta por turno). Dessa maneira, em uma mão com $h$ cartas, temos
$h+1$ possibilidades, cada uma com probabilidade
\begin{equation} \label{eq:stateprob} P_h(i) = \frac{\binom{t}{i}\binom{60 - t}{h - i}}{\binom{60}{h}}, \ \
i = 0,\ldots, h, \end{equation} onde $t$ é o número de terrenos no deck e $i$ o
número de terrenos
na mão. Assim, a cada nível, como é mostrado na figura \ref{mulligan}, o
agente deve decidir entre realizar um \textit{mulligan}
(denotado por $M$), que resultará em uma mão com $h-1$ cartas, sendo $i$
terrenos com probabilidade
 \[ P_{h - 1}(i) = \frac{\binom{t}{i}\binom{60 - t}{h - 1 -
i}}{\binom{60}{h - 1}}, \ \  i = 0,\ldots, h - 1\]
 ou manter a mão (denotado por $K$), o que termina o problema, seguido
(para $h < 7$) de uma ação \textit{scry}, que permite uma pequena
``filtragrem''
 do deck para os jogadores que já acumularam alguma desvantagem (em
relação ao número de cartas na mão) no processo. Por fim, a decisão
também é influenciada
 pela informação de quem é o jogador inicial, pois ele tem virtualmente
uma carta a menos, dado que no primeiro turno do jogo não se compra uma
carta.

\subsection{Aprendizado por Reforço}


Dada a estrutura bem conhecida do problema, o Aprendizado por Reforço é
uma técnica que pode ser aplicada
para que o agente determine um comportamento (política) para alcançar
uma mão ``jogável'' para uma partida de \textit{Magic}.

O problema pode ser representado como uma série de estados, cada um
representando uma configuração de quantidade de cartas na mão
e quantidade de terrenos presentes entre essas cartas. Em cada mão, o
agente pode decidir ou manter a mão e ``resgatar''
os pontos que aquela configuração representa, ou realizar um
\textit{mulligan}, que resultará em uma mão com uma carta a menos e
distribuição de terrenos descrita na subseção anterior. Assim, depois de
várias iterações, o agente irá encontrar uma política $\pi$
que maximize a utilidade total $U(\pi) = \sum U^\pi(s)$, onde $U^\pi(s)$
é a função-utilidade do estado $s$. Usando a equação de
Bellman, temos \[ U^\pi(s) = R(s) + \gamma\sum\limits_{s'}P(s' | s,
\pi(s))U^\pi(s'),\] ou seja, a utilidade de cada estado é a recompensa
$R(s)$
do estado somada à soma da utilidade de cada estado sucessor $s'$ vezes
a probabilidade de chegar em $s'$ a partir de $s$ e da política adotada
em $s$ (esta soma é multiplicada por um fator de desconto $\gamma$).
Assim, temos que cada estado pode ser representado pelo par $(h, i)$ e a função de Bellman, reescrita:
\begin{equation} \label{eq:mullbellman}U^\pi(h, i) = R(h, i) + \gamma \sum\limits_{i' = 0}^{h - 1}P(h-1, i' | h , i )U^\pi(h - 1, i').\end{equation}
A probabilidade de cada estado sucessor $(h- 1, i')$ é dada pela equação \ref{eq:stateprob}:
\[P(h-1, i' | h, i) = P_{h-1}(i') = \frac{\binom{t}{i'}\binom{60 - t}{h - 1 -
i'}}{\binom{60}{h - 1}}, \ \  i = 0,\ldots, h - 1.\]

Por fim, ainda precisamos determinar as recompensas $R(h, i)$ e o fator de desconto $\gamma$.
Uma boa relação $R(h,i)$ de recompensas, por experiências pessoais, deve seguir as seguintes restrições:
\begin{itemize}
  \item Com $h = h_0$, as recompensas (se existem) devem seguir a seguinte relação:
  \begin{equation} \label{eq:idealhand}R(h_0, 3) > R(h_0, 2) > R(h_0, 4) > R(h_0, 5) > R(h_0, 1) > R(h_0, 6) > R(h_0, 7) \gg R(h_0, 0),\end{equation}
  pois na média, uma mão ideal tem 3 terrenos, em segundo lugar 2 terrenos, etc. Uma mão sem terrenos é considerada \textbf{muito} ruim.
  \item $R(h, i) > R(h - 1, i), \forall (h, i):$ uma mão com o mesmo número de terrenos e uma carta a menos deve sempre ser considerada pior do que a alternativa.
  \item $R(h, i) > R(h - 1, i - 1), \forall (h, i):$ uma mão com o mesmo número de não-terrenos e uma carta a menos deve sempre ser considerada pior do que a alternativa.
\end{itemize}

\begin{table}[!b]

\parbox{.45\linewidth}{
\centering
\vspace{0.2cm}
\begin{tabular}{l|r}
$i$  & $r(i)$ \\ \hline
0 & -1000  \\
1 & -4     \\
2 & 0      \\
3 & 1      \\
4 & -1     \\
5 & -3     \\
6 & -5     \\
7 & -7
\end{tabular}
\caption{recompensas-base para $i = 0, \ldots, 7$}
\label{tab:ri}
}
\hfill
\parbox{.45\linewidth}{
\centering
\begin{tabular}{l|rrrrrrrr}
  & 0     & 1  & 2  & 3  & 4  & 5  & 6  & 7  \\ \hline
7 & -979  & 17 & 21 & 22 & 20 & 18 & 16 & 14 \\
6 & -982  & 14 & 18 & 19 & 17 & 15 & 13 &    \\
5 & -985  & 11 & 15 & 16 & 14 & 12 &    &    \\
4 & -988  & 8  & 12 & 13 & 11 &    &    &    \\
3 & -991  & 5  & 9  & 10 &    &    &    &    \\
2 & -994  & 2  & 6  &    &    &    &    &    \\
1 & -997  & -1 &    &    &    &    &    &    \\
0 & -1000 &    &    &    &    &    &    &
\end{tabular}
\caption{recompensas de cada estado calculadas com $R(i,h) = r(i) + 3h$}
\label{tab:Rhi}
}
\end{table}

Dessa maneira, chegamos a uma fórmula geral, $R(h, i) = r(i) + \alpha h$, onde $r(i)$ assume valores pré-determinados de acordo com (\ref{eq:idealhand}). Para $\alpha = 3$ e $r(i)$ determinado pela tabela \ref{tab:ri}, podemos atribuir as recompensas de acordo com a tabela \ref{tab:Rhi}










\end{document}
