\chapter{Implementação do Cliente}
Para que fosse possível que implementássemos um agente de inteligência artificial que
jogasse uma versão simplificada de Magic, era essencial que conhecêssemos em detalhe a
plataforma onde o jogo estaria rodando, e para isso implementamos do zero uma versão do
jogo com as especifidades desejadas usando a linguagem Python.

O programa é composto de quatro classes principais: \texttt{Game}, \texttt{Player}, \texttt{Card}
e \texttt{Permanent}, representando o Jogo, Jogadores, Cartas e Permanentes (como são tratadas as
cartas de Terreno e Criatura uma vez que estão em jogo) respectivamente. A seguir vamos falar de
cada uma destas classes entrando em detalhes nas principais características.

\section{\texttt{class Card:}}
A classe \texttt{Card} representa as cartas do jogo. Os objetos que estarão presentes nas mãos,
decks e cemitérios dos jogadores serão classes que herdam desta classe. Seus atributos representam
características presentes em todas as cartas. Para exemplificar os atributos, usaremos duas cartas:
Martelo Vulcânico (Volcanic Hammer) e Anjo da Misericórdia (Angel of Mercy).

\begin{figure}[!h]
    \centering
    \begin{minipage}{0.45\textwidth}
        \centering
        \includegraphics[width=0.5\textwidth]{picstcc/VolcanicHammer.jpg}
        \caption{Martelo Vulcânico (Feitiço)}
    \end{minipage}\hfill
    \begin{minipage}{0.45\textwidth}
        \centering
        \includegraphics[width=0.5\textwidth]{picstcc/angelOfMercy.jpg}
        \caption{Anjo da Misericórdia (Criatura)}
    \end{minipage}
\end{figure}

Os atributos a seguir são comuns a todos os tipos de carta, e por isso estão presentes na
classe Card:

\begin{itemize}
  \item\texttt{name}: Uma string representando o nome da carta, por exemplo ``Volcanic Hammer''
  ou ``Angel of Mercy''.
  \item\texttt{cost}: Uma string representando o custo de mana da carta, usando a mesma notação
  presente nas cartas, com W, U, B, R e G representando os símbolos de mana das cores Branco,
  Azul, Preto, Vermelho e Verde, respectivamente. No caso de Volcanic Hammer, o custo é ``1R''
  e no de Angel of Mercy é ``4W''.
  \item\texttt{supertype}: Uma string representando o supertipo da carta. No nosso programa,
  o único supertipo que irá aparecer é ``Basic'', que identifica os terrenos básicos, mas essa
  informação não é relevante para o programa (um terreno ser básico significa que um deck pode
  conter qualquer quantidade deste terreno, ao invés do limite normal de quatro cópias por carta).
  Estamos representando essa informação principalmente por formalidade para que o programa siga a
  mesma estrutura de tipos descrita nas regras do jogo. Nem Volcanic Hammer nem Angel of Mercy
  possuem supertipos, portanto este atributo é representado pela string vazia.
  \item\texttt{ctype}: Uma string que representa o tipo da carta. Angel of Mercy é do tipo
  ``Creature'' e Volcanic Hammer é do tipo ``Sorcery''. Diferentemente de supertipo ou subtipo,
  este é um atributo que nunca estará vazio em uma carta.
  \item\texttt{subtype}: Uma string que representa o subtipo da carta. O subtipo serve para que
  algumas cartas tenham uma informação adicional que possa ser usada para ações dentro do jogo.
  Volcanic Hammer não tem subtipo, enquanto o subtipo de Angel of Mercy é ``Angel''.
  \item\texttt{text}: Uma string representando o texto da carta. Essa string serve somente para
  interface com o jogador. O texto de Volcanic Hammer é ``Volcanic Hammer deals 3 damage to target
  creature or player.'' enquanto o de Angel of Mercy é ``Flying. When Angel of Mercy enters the
  battlefield, you gain 3 life. 3/3''.
  \item\texttt{targets}: Uma lista contendo tuplas com as possibilidades de alvo que a carta
  pode ter. Angel of Mercy não tem alvos, e portanto sua lista de alvos é vazia. Volcanic Hammer
  pode dar alvo em criaturas ou jogadores, então sua lista de alvos é
  \texttt{[(``OwnCreature'', ``OpponentCreature'', ``Player'')]}. Separamos criaturas  entre
  criaturas do mesmo dono da carta ou criaturas dos oponentes pois há cartas que só podem ter
  um destes conjuntos como alvo, facilitando a implementação.
  \item\texttt{owner}: O jogador dono da carta. Este atributo é do tipo \texttt{Player}.
\end{itemize}
Para cada carta com nome diferente, há uma classe que herda da classe \texttt{Card} e implementa
as especificidades da carta. Vejamos o código da classe \texttt{VolcanicHammer}, que implementa
a carta Volcanic Hammer:
\begin{figure}
  \begin{Verbatim}[commandchars=\\\{\}]
\PY{k}{class} \PY{n+nc}{VolcanicHammer}\PY{p}{(}\PY{n}{Card}\PY{p}{)}\PY{p}{:}
    \PY{k}{def} \PY{n+nf+fm}{\PYZus{}\PYZus{}init\PYZus{}\PYZus{}}\PY{p}{(}\PY{n+nb+bp}{self}\PY{p}{,} \PY{n}{owner}\PY{p}{)}\PY{p}{:}
        \PY{n+nb+bp}{self}\PY{o}{.}\PY{n}{name} \PY{o}{=} \PY{l+s+s2}{\PYZdq{}}\PY{l+s+s2}{Volcanic Hammer}\PY{l+s+s2}{\PYZdq{}}
        \PY{n+nb+bp}{self}\PY{o}{.}\PY{n}{cost} \PY{o}{=} \PY{l+s+s2}{\PYZdq{}}\PY{l+s+s2}{1R}\PY{l+s+s2}{\PYZdq{}}
        \PY{n+nb+bp}{self}\PY{o}{.}\PY{n}{supertype} \PY{o}{=} \PY{l+s+s2}{\PYZdq{}}\PY{l+s+s2}{\PYZdq{}}
        \PY{n+nb+bp}{self}\PY{o}{.}\PY{n}{ctype} \PY{o}{=} \PY{l+s+s2}{\PYZdq{}}\PY{l+s+s2}{Sorcery}\PY{l+s+s2}{\PYZdq{}}
        \PY{n+nb+bp}{self}\PY{o}{.}\PY{n}{subtype} \PY{o}{=} \PY{l+s+s2}{\PYZdq{}}\PY{l+s+s2}{\PYZdq{}}
        \PY{n+nb+bp}{self}\PY{o}{.}\PY{n}{text} \PY{o}{=} \PY{l+s+s2}{\PYZdq{}}\PY{l+s+s2}{Volcanic Hammer deals 3 damage to target creature or player.}\PY{l+s+s2}{\PYZdq{}}
        \PY{n+nb+bp}{self}\PY{o}{.}\PY{n}{targets} \PY{o}{=} \PY{p}{[}\PY{p}{[}\PY{l+s+s2}{\PYZdq{}}\PY{l+s+s2}{OwnCreature}\PY{l+s+s2}{\PYZdq{}}\PY{p}{,} \PY{l+s+s2}{\PYZdq{}}\PY{l+s+s2}{OpponentCreature}\PY{l+s+s2}{\PYZdq{}}\PY{p}{,} \PY{l+s+s2}{\PYZdq{}}\PY{l+s+s2}{Player}\PY{l+s+s2}{\PYZdq{}}\PY{p}{]}\PY{p}{]}
        \PY{n+nb+bp}{self}\PY{o}{.}\PY{n}{owner} \PY{o}{=} \PY{n}{owner}

    \PY{k}{def} \PY{n+nf}{effect}\PY{p}{(}\PY{n+nb+bp}{self}\PY{p}{,} \PY{n}{game}\PY{p}{,} \PY{n}{targets}\PY{p}{)}\PY{p}{:}
        \PY{n}{targets}\PY{p}{[}\PY{l+m+mi}{0}\PY{p}{]}\PY{o}{.}\PY{n}{takeDamage}\PY{p}{(}\PY{l+m+mi}{3}\PY{p}{)}
\end{Verbatim}

\end{figure}
Podemos ver que a classe tem o método \texttt{effect}, que implementa o efeito da carta,
fazendo com que o alvo escolhido sofra três pontos de dano. Vejamos agora a implementação
da carta Angel of Mercy:
\begin{figure}
  
\begin{Verbatim}[commandchars=\\\{\}]
\PY{k}{class} \PY{n+nc}{AngelofMercy}\PY{p}{(}\PY{n}{Card}\PY{p}{)}\PY{p}{:}
    \PY{k}{def} \PY{n+nf+fm}{\PYZus{}\PYZus{}init\PYZus{}\PYZus{}}\PY{p}{(}\PY{n+nb+bp}{self}\PY{p}{,}\PY{n}{owner}\PY{p}{)}\PY{p}{:}
        \PY{n+nb+bp}{self}\PY{o}{.}\PY{n}{name} \PY{o}{=} \PY{l+s+s2}{\PYZdq{}}\PY{l+s+s2}{Angel of Mercy}\PY{l+s+s2}{\PYZdq{}}
        \PY{n+nb+bp}{self}\PY{o}{.}\PY{n}{cost} \PY{o}{=} \PY{l+s+s2}{\PYZdq{}}\PY{l+s+s2}{4W}\PY{l+s+s2}{\PYZdq{}}
        \PY{n+nb+bp}{self}\PY{o}{.}\PY{n}{supertype} \PY{o}{=} \PY{l+s+s2}{\PYZdq{}}\PY{l+s+s2}{\PYZdq{}}
        \PY{n+nb+bp}{self}\PY{o}{.}\PY{n}{ctype} \PY{o}{=} \PY{l+s+s2}{\PYZdq{}}\PY{l+s+s2}{Creature}\PY{l+s+s2}{\PYZdq{}}
        \PY{n+nb+bp}{self}\PY{o}{.}\PY{n}{subtype} \PY{o}{=} \PY{l+s+s2}{\PYZdq{}}\PY{l+s+s2}{Angel}\PY{l+s+s2}{\PYZdq{}}
        \PY{n+nb+bp}{self}\PY{o}{.}\PY{n}{text} \PY{o}{=} \PY{l+s+s2}{\PYZdq{}}\PY{l+s+s2}{Flying. When Angel of Mercy enters the battlefield, you gain 3 life. 3/3}\PY{l+s+s2}{\PYZdq{}}
        \PY{n+nb+bp}{self}\PY{o}{.}\PY{n}{abilities} \PY{o}{=} \PY{p}{[}\PY{l+s+s2}{\PYZdq{}}\PY{l+s+s2}{Flying}\PY{l+s+s2}{\PYZdq{}}\PY{p}{]}
        \PY{n+nb+bp}{self}\PY{o}{.}\PY{n}{targets} \PY{o}{=} \PY{p}{[}\PY{p}{]}
        \PY{n+nb+bp}{self}\PY{o}{.}\PY{n}{owner} \PY{o}{=} \PY{n}{owner}
        \PY{n+nb+bp}{self}\PY{o}{.}\PY{n}{power} \PY{o}{=} \PY{l+m+mi}{3}
        \PY{n+nb+bp}{self}\PY{o}{.}\PY{n}{tou} \PY{o}{=} \PY{l+m+mi}{3}

    \PY{k}{def} \PY{n+nf}{effect}\PY{p}{(}\PY{n+nb+bp}{self}\PY{p}{,} \PY{n}{game}\PY{p}{,} \PY{n}{targets}\PY{p}{)}\PY{p}{:}
        \PY{n+nb+bp}{self}\PY{o}{.}\PY{n}{owner}\PY{o}{.}\PY{n}{gainLife}\PY{p}{(}\PY{l+m+mi}{3}\PY{p}{)}
\end{Verbatim}

\end{figure}
Além dos atributos que citamos anteriormente, Angel of Mercy tem três atributos a mais por
ser do tipo Criatura: \texttt{abilities}, uma lista com as habilidades da criatura;
\texttt{power}, o poder da criatura; e \texttt{tou}, a resistência (toughness) da criatura.
Podemos também ver o efeito da carta, que concede três pontos de vida a seu controlador.

\section{\texttt{class Permanent:}}
A classe \texttt{Permanent} serve para representar as cartas uma vez que foram jogadas
e estão no campo de batalha. Seus atributos são os seguintes:
\begin{itemize}
  \item\texttt{card}: A carta que criou a permanente. Ela contém informações relevantes
  para a construção do objeto e é ela, e não a permanente, que será colocada no cemitério
  caso deixe o campo de batalha.
  \item\texttt{abilities}: Uma cópia do atributo \texttt{abilities} da carta. É necessário
  que esse atributo seja armazenado separadamente ao da carta, pois uma vez em jogo, a
  permanente pode ganhar ou perder habilidades.
  \item\texttt{ctype}: O tipo da permanente, conforme descrito na carta que a criou.
  \item\texttt{owner}: O jogador dono da permanente, que será o mesmo que possúi a carta.
  \item\texttt{controller}: O jogador controlador da permanente. Normalmente ele será o dono,
  mas há efeitos que mudam o jogador que controla uma permanente.
  \item\texttt{tapped}: Um atributo booleano que indica se a permanente está virada ou
  desvirada.
  \item\texttt{sick}: Um atributo booleano que indica se a permanente está
  \textit{enjoada}\footnote{Permanentes entram em jogo com \textit{enjoo de invocação}, o que
  indica que caso ela seja uma criatura, não poderá atacar. O enjoo de uma permanente acaba no
  começo do turno de seu controlador.} ou não.
  \item\texttt{destroyed}: Um atributo booleano que indica se a permanente foi destruída ou não.
  Permanentes destruídas são colocadas no cemitério de seu controlador durante as checagens do jogo.
  \item\texttt{ID}: Um inteiro que serve como identificador único entre permanentes. Ele serve,
  por exemplo, para diferenciar permanentes que tenham sido criadas a partir de cartas iguais.
\end{itemize}
Estes atributos são alterados a partir dos métodos da classe, que não entraremos em detalhes.

Há duas classes que herdam da classe \texttt{Permanent}: \texttt{Land} e \texttt{Creature}.
\texttt{Land} é a classe que representa terrenos. A única diferença desta classe para a classe
mãe é que os métodos de virar e desvirar manipulam um atributo do jogador controlador: a quantidade
de terrenos desvirados controlados por aquele jogador. A classe \texttt{Creature} tem alguns
atributos a mais:
\begin{itemize}
  \item\texttt{power}: Um inteiro que representa o poder da criatura.
  \item\texttt{tou}: Um inteiro que representa a resistência da criatura.
  \item\texttt{curPower}: Um inteiro que representa o poder atual da criatura. Existem efeitos
  que modificam o poder com duração limitada a um turno. Nestes casos, o poder volta a ser o
  poder armazenado no atributo \texttt{power} quando o próximo turno começar.
  \item\texttt{curTou}: Um inteiro que representa a resistência atual da criatura. Análogo ao
  atributo \texttt{curPower} relativo à resistência da criatura.
  \item\texttt{attacking}: Um atributo booleano que representa se a criatura está atacando
  ou não.
  \item\texttt{blocking}: Um atributo booleano que representa se a criatura está bloqueando
  ou não.
  \item\texttt{damage}: Um inteiro que armazena a quantidade de dano sofrida pela criatura até
  o presente momento. No início de cada turno, este atributo volta a ter o valor $0$.
  \item\texttt{currentAbilities}: Análogo ao atributo \texttt{curPower} para as habilidades.
\end{itemize}
Criaturas têm alguns métodos especiais que possibilitam com que elas realizem ações dentro
do jogo, como atacar ou causar dano, mas não entraremos em detalhes.

\section{\texttt{class Player:}}
A classe \texttt{Player} representa um jogador na partida. Seus atributos são:
\begin{itemize}
  \item\texttt{name}: Uma string representando o nome do jogador. Não tem nenhuma função
  dentro do jogo. Serve para identificar os jogadores na saída.
  \item\texttt{life}: Um inteiro representando o valor atual do total de pontos de vida
  do jogador.
  \item\texttt{hand}: Uma lista de objetos do tipo \texttt{Card}, representando as cartas
  na mão do jogador.
  \item\texttt{library}: Uma lista de objetos do tipo \texttt{Card}, representando as cartas
  no deck\footnote{Dentro do jogo, o deck recebe o nome de \textit{Grimório}, ou \textit{Library},
  em inglês.} do jogador.
  \item\texttt{lose}: Um atributo booleano para identificar se o jogador perdeu o jogo. Este
  atributo é utilizado pela classe \texttt{Game} para decidir se o jogo acabou ou não.
  \item\texttt{ID}: Um inteiro representando um identificador único para o jogador, de maneira
  similar à que acontece com as permanentes.
  \item\texttt{creatures}: Uma lista de objetos da classe \texttt{Creature} com todas as
  criaturas controladas pelo jogador.
  \item\texttt{lands}: Uma lista de objetos da classe \texttt{Land} com todos os terrenos
  controlados pelo jogador.
  \item\texttt{active}: Um atributo booleano que indica se o jogador é o jogador ativo.
  \item\texttt{graveyard}: Uma lista de objetos da classe \texttt{Card} com todas as cartas
  no cemitério do jogador.
  \item\texttt{untappedLands}: Um inteiro representando a quantidade de terrenos desvirados que
  o jogador controla.
  \item\texttt{landDrop}: Um atributo booleano indicando se o jogador já jogou um terreno no
  turno atual.
\end{itemize}

\section{\texttt{class Game:}}
A classe \texttt{Game} representa um jogo em andamento. Devido à complexidade da classe, entrar
em detalhes iria mudar o foco do trabalho, então não iremos descrever todos os métodos e atributos
da classe. Há um método chamado \texttt{turnRoutine()} que é chamado repetidamente até que o jogo
termine. Ele executa as ações iniciais e finais do turno (como fazer com que o jogador ativo compre
uma carta no começo do turno e limpar o dano das criaturas no final do turno) e chama os métodos de
fase principal e de combate, que permitem que os jogadores joguem as cartas da mão, ataquem e
bloqueiem.

\vskip1ex

Além disso há alguns métodos para definir quais são as ações legais para um jogador e para realizar
as chamades \texttt{ações baseadas no estado}, que são ações performadas pelo jogo sempre que uma
certa condição baseada no estado do jogo se torna verdade (por exemplo, destruir criaturas com dano
igual ou superior à resistência, fazer com que jogadores com $0$ ou menos pontos de vida percam o
o jogo, e assim por diante). O método que cuida disso é \texttt{checkSBA()} (de ``state based actions''),
chamado pela classe \texttt{Game} nos seguintes momentos:
\begin{itemize}
  \item No início e fim de cada fase.
  \item Após toda ação de Fase Principal.
  \item Durante etapas de atribuição de dano no Combate.
\end{itemize}
